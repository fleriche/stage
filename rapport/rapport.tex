\documentclass[11pt,twoside]{scrreprt}
\usepackage[utf8]{inputenc} 	%package pour le français sous ubuntu : à vous d'adapter
\usepackage[french]{babel}	%pour le français
\usepackage[T1]{fontenc}	%pour les polices
\usepackage{amsmath}		%pour des maths
\usepackage{amsfonts}		%pour des maths
\usepackage{amssymb}		%pour des maths
\usepackage{graphicx}		%pour les inclusions de graphiques
\usepackage{times}		%choix personnel de police : Times New Roman 
\usepackage{lscape}		%si vous voulez des images en paysage
\usepackage{hyperref}		%pour les liens croisés à travers le fichier .pdf
\usepackage{fancyhdr}		%pour les marges
\usepackage{float, caption}	%pour le positionnement et légende des images
\usepackage{color}		%pour la couleur dans le code
\usepackage{listings}		%pour mettre du code, plutôt en Annexe
\usepackage{lastpage}		%pour les références aux pages
\usepackage{epic,eepic}		%pour le positionnement de l'image de garde
\usepackage{wrapfig}		%pour les images sur le coté dans le texte
\usepackage{calc,ifthen,xspace}	%pour redéfinir les espaces et distances
\usepackage{scrlfile}
\PreventPackageFromLoading{fp}
\usepackage[final]{pdfpages} % Pour inclure un pdf
\usepackage[titletoc]{appendix} % Pour les annexes
\usepackage{glossaries} % Pour le lexique
\usepackage{tabularx}
\usepackage{diagbox}
\ResetPreventPackageFromLoading
\usepackage{array}
\pagestyle{fancy}
\AddThinSpaceBeforeFootnotes 
\FrenchFootnotes 
\makeglossaries
\usepackage{lmodern}
\usepackage{textcomp}
\usepackage{xspace}
\usepackage{listingsutf8}
\usepackage{xcolor}
\usepackage{afterpage}
\usepackage{url}
\usepackage[top=2.1cm,bottom=2.3cm,left=2cm,right=2cm]{geometry}
\usepackage{multirow}
\usepackage{verbatim}

% Pour sommaire cliquable 

\hypersetup{
dvips,
backref=true, %permet d'ajouter des liens dans...
pagebackref=true,%...les bibliographies
hyperindex=true, %ajoute des liens dans les index.
colorlinks=false, %colorise les liens
breaklinks=true, %permet le retour à la ligne dans les liens trop longs
urlcolor= blue, %couleur des hyperliens
linkcolor= blue, %couleur des liens internes
bookmarks=blue, %créé des signets pour Acrobat
bookmarksopen=true} 

\definecolor{hellgelb}{rgb}{1,1,0.8} % couleur pour le code
\definecolor{colKeys}{rgb}{0,0,1}
\definecolor{colIdentifier}{rgb}{0,0,0}
\definecolor{colComments}{rgb}{0,0.5,0}
\definecolor{colString}{rgb}{0.62,0.12,0.94}
\definecolor{INSA_GM}{cmyk}{0.6,0,0,0} % et la page de garde
\definecolor{INSA_GRIS}{cmyk}{0.7,0.6,0.5,0.3}
\definecolor{INSA_BLEU}{cmyk}{1,0.9,0.1,0}
\definecolor{mygreen}{rgb}{0,0.6,0}
\definecolor{mygray}{rgb}{0.5,0.5,0.5}
\definecolor{mymauve}{rgb}{0.58,0,0.82}
\definecolor{mygrayblack}{rgb}{0.32,0.32,0.32}
\colorlet{punct}{red!60!black}
\definecolor{background}{HTML}{EEEEEE}
\definecolor{delim}{RGB}{20,105,176}
\colorlet{numb}{magenta!60!black}

\usepackage{templateINSA}
\initINSA

% Titre centre
\renewcommand\infoBig{Rapport de stage ingénieur}
\renewcommand\infoSmall{}

% Titre bas 
\title{Sous titre}
\renewcommand\soustitre{INSA de Rouen}

% Auteurs
\author{ 
	\textbf{Étudiants:} Florian \bsc{Leriche}\\
	\textbf{Université:} INSA de Rouen\\
	\textbf{Date:} 13/02/2017 - 11/08/2017\\
	}

\begin{document}
\thispagestyle{empty}
% titleINSA : Page de garde 
% #1 : descendre le titre du milieu (en mm)
% #2 : lien de l'image de fond
% #3 : décalage sur X de l'image de fond (en mm)
% #4 : décalage sur Y de l'image de fond (en mm)
% #5 : largeur de l'image de fond de #5 (en mm)
% #6 : Crédit de l'image de fond
\titleINSA{10}{}{-10}{100}{240}{Image : \href{http://www.google.fr/}{\color{white}{http://www.google.fr/}}}

\thispagestyle{empty}

%-----------------------------------------------------------------------------------------------------------------
%-----------------------------------------------------------------------------------------------------------------
	\tableofcontents
	\thispagestyle{empty}
%-----------------------------------------------------------------------------------------------------------------
%-----------------------------------------------------------------------------------------------------------------

\newpage

\chapter*{Remerciements}
	\addcontentsline{toc}{chapter}{Remerciements}
	\setcounter{page}{1}
	\setcounter{tocdepth}{1}
Je tiens en premier lieu à remercier Sopra Steria Group de m’avoir accueilli durant ce stage et de m’avoir donné la possibilité à travers mon stage d’acquérir de l’expérience dans un domaine qui me tient à cœur, de découvrir un environnement de travail en adéquation avec mes études et m’ayant permis de mettre en pratique les connaissances acquises au cours de ma formation théorique. \\

Je remercie, en particulier, mes tuteurs de stage Cyril Sacenda et Berthaud Laurent pour m’avoir encadré et sans qui ce stage n’aurait pas pu avoir lieu. Je les remercie eux ainsi que l'équipe RH pour toutes les démarches qu’ils ont pu entreprendre à mon égard. \\

Je remercie également grandement les développeurs Bouhamyd Mohamed et Azzerrifi Laamrani Abdellatif côté Neuflize OBC et Rinfray Corentin et Madih Jaafar côté BP1818 ainsi que les architectes Rudy Jansem, Barillet Cyril et N'Takpe Jocelyn pour lesquels j’ai travaillé directement, pour m’avoir conseillé et aidé lorsque j’en avais besoin. Je suis extrêmement reconnaissant du temps qu’ils ont pu m’accorder et de l’enrichissement qu’ils ont pu m’apporter en partageant leur connaissance avec moi. \\

Enfin, je remercie tout le personnel avec lequel j’ai eu l’occasion de travailler et qui m’ont beaucoup apporté et aidé lors de mon stage. \\

En outre, je remercie aussi la communauté architecte du marché Banque et finance pour les Brown Bag Lunch (présentation de technologies et retour d'expérience sur l'heure du repas) ainsi que l'équipe commerciale pour m'avoir fait intervenir dans le cadre d'un retour d'expérience à la soirée du Digital Banking.

%-----------------------------------------------------------------------------------------------------------------
%-----------------------------------------------------------------------------------------------------------------
	\chapter*{Introduction}				% garde la même mise en page qu'un chapitre mais ne le numérote pas
	\addcontentsline{toc}{chapter}{Introduction}	% permet de l'avoir dans le sommaire

\paragraph{}
Le présent rapport détail le travail qui s’inscrit dans le cadre de la réalisation d’un stage ingénieur effectué par les élèves ingénieurs de l’Institut National des Sciences Appliquées de Rouen. Celui-ci s’est déroulé via la participation à deux missions distinctes pour le compte de la société de services \textit{Sopra Steria} du 13 février au 11 août 2017; suite à une convention tripartite signée entre le département Architecture des Systèmes d’Information de L’INSA, l’entreprise d’accueil et moi-même. Pendant ce stage, j’ai eu l’occasion de participer à plusieurs projets qui m’ont permis d’appréhender le métier d’ingénieur en informatique, d’acquérir de l’expérience ainsi que de m’épanouir aussi bien dans le plan personnel que professionnel.

\paragraph{}
La première mission à laquelle j'ai pris part s'est déroulée du 13 février 2017 jusqu'à la fin de mon stage, pour le compte du client \textit{Neuflize OBC}. La seconde mission a débuté le 29 mars 2017 pour la banque privée \textit{BP1818} et s'est déroulée en parallèle de la première jusqu'à la fin du stage. La répartition du temps de travail était la suivante :
\begin{itemize}
	\item Lundi, mardi et vendredi chez le client \textit{BP1818}
	\item Mercredi et jeudi chez le client \textit{Neuflize OBC}
\end{itemize}

\paragraph{}
Sopra Steria est une entreprise de services du numérique et l'un des leader européens dans la transformation numérique. Ainsi, l'objectif premier de mon stage a été d'accompagner certains des clients banques privées de Sopra Steria dans leur transformation digital en travaillant à la fois sur la réalisation et l'industrialisation d'une application mobile à destination des clients de la banque Neuflize ainsi que sur la mise en place d'une application web à destination des banquiers travaillant chez BP1818.

\paragraph{}
Dans un premier temps, je présenterai l'entreprise d'accueil et les client pour lesquels j'ai travaillé, leurs domaines d’activités, leurs origines, leurs personnel ainsi que leurs locaux.
Dans un second temps, je détaillerai de manière précise les sujets des missions qui m’ont été confiées au sein des équipes.
Enfin, je décrirai en profondeur le déroulement de mon stage ainsi que les différents travaux que j’ai accompli et les conditions dans lesquelles je les ai réalisés. 

\paragraph{}
Pour des raisons de clarté et de cohérence, ce rapport présentera deux parties distinctes concernant le déroulement du stage, chacune ayant pour objectif de décrire le travail réalisé chez les clients cités plus haut et suivant le même schéma.
%-----------------------------------------------------------------------------------------------------------------
%-----------------------------------------------------------------------------------------------------------------


%-----------------------------------------------------------------------------------------------------------------
%-----------------------------------------------------------------------------------------------------------------	
\chapter{Spécifications}

	\input{src/specification}
%-----------------------------------------------------------------------------------------------------------------
%-----------------------------------------------------------------------------------------------------------------

%-----------------------------------------------------------------------------------------------------------------
%-----------------------------------------------------------------------------------------------------------------
\chapter{Conception}
	
	\input{src/conception}
%-----------------------------------------------------------------------------------------------------------------
%-----------------------------------------------------------------------------------------------------------------
	
%-----------------------------------------------------------------------------------------------------------------
%-----------------------------------------------------------------------------------------------------------------	
\chapter{Implémentation et résultats}

	\input{src/implementation}
%-----------------------------------------------------------------------------------------------------------------
%-----------------------------------------------------------------------------------------------------------------
	
%-----------------------------------------------------------------------------------------------------------------
%-----------------------------------------------------------------------------------------------------------------
\chapter*{Conclusion et perspectives} %même mise en page chapitre mais numérote pas
	\addcontentsline{toc}{chapter}{Conclusion et perspectives} %ajout au sommaire
	
	\newenvironment{changemargin}[2]{%
\begin{list}{}{%
\setlength{\topsep}{0pt}%
\setlength{\leftmargin}{#1}%
\setlength{\rightmargin}{#2}%
\setlength{\listparindent}{\parindent}%
\setlength{\itemindent}{\parindent}%
\setlength{\parsep}{\parskip}%
}%
\item[]}{\end{list}}

\begin{changemargin}{-1cm}{-1cm}

	Ce stage a été, pour moi, extrêmement enrichissant de par la richesse des tâches que j'ai pu effectuer. En effet, j'ai eu la chance de pouvoir travailler à mi-temps sur deux projets importants de l'agence 512, plus grosse agence sur secteur bancaire de Sopra Steria. Ces projets digitaux sont actuellement des modèles ainsi que des portes d'entrées vers de nouveaux projets et clients voulant franchir le pas de la transformation numérique. Sur ces projets j'ai pu à chaque fois travailler au sein d'équipes de développement dynamiques intégrées dans leur environnement et centrée autour de la collaboration pour la réussite d'un projet digital dans son ensemble et non pas pour la simple réalisation d'un produit à vendre. Aussi, de par l'utilisation des méthodes agiles et des relations ainsi créées j'ai pu sortir du cadre développeur classique afin de m'entretenir directement avec le client, que ce soit pour revoir des spécifications et le conseiller comme sur le sujet des dashboards, pour des recettes ou encore pour que celui-ci m'éclaircisse son point de vue et son besoin. Cela permet d'avoir une idée nettement plus claire de ses attentes mais aussi des contraintes auxquelles il doit faire face et donc d'être force de proposition dans le but de lui venir en aide. Il s'agit là d'une véritable plus-value pour mon insertion professionnelle puisque j'ai eu la chance de travailler et d'interagir non seulement avec des développeurs mais aussi avec des architectes, des équipes fonctionnelles, de pilotage, de conduite du changement, des commerciaux et bien sûr des équipes métiers. \\
	
	Durant mon stage j'ai eu la chance d'être intégré à la "communauté architecture" de la division banque et finance de Sopra Steria dans laquelle une veille technologique est assurée et des brown bag lunch (BBL) sont organisés. Il s'agit de présentations sur des sujets technologiques variés effectuées sur l'heure du repas par des développeurs ou des architectes membres de la communauté. Cela apporte une fois encore un côté humain favorisant les relations tout en permettant de rester centré autour des métiers de l'information. Les architectes essaient de présenter les dernières technologies et n'hésitent pas à les employer dans leurs projets actuels. Par exemple, chez Neuflize OBC, une architecture microservices a été mise en place, une première dans la division bancaire Sopra Steria. Le succès de cette dernière a pu être transmis par le biais de la communauté et une étude est menée en ce moment par des développeurs de chez BForBank pour l'utiliser sur l'un de leur projet. De plus, les outils utilisés sont ceux de la stack Netflix, récents et open-source. Du côté de BP1818, nous avons recours à Angular2 pour construire notre frontend et nous avons même migré sur Angular4 sorti en version stable au mois de mars 2017, puis Angular5 en juillet. \\
	
	Un autre point fort que j'ai pu relever durant ce stage concerne la richesse des travaux que j'ai pu réaliser. Du côté de BP1818 j'ai pu effectuer du développement aussi bien sur la partir backend que sur la partir frontend. Une certaine liberté m'a été laissé puisque j'ai souvent pu réaliser mes tâches de manière autonome tant au niveau conception que développement bien qu'étant sous la supervision de mon tuteur en cas de nécessité. Chez Neuflize OBC, je n'ai effectué que peu de développement mais j'ai pu m'intéresser à la partie tests fonctionnels de par leur réalisation et la mise en place d'outils pour les automatiser et faire gagner du temps à l'équipe. J'ai aussi pu mettre en place du monitoring applicatif, sujet durant lequel je pouvais moi-même organiser des points avec les métiers. De plus, une fois mes sujets terminés et avant mon passage à temps plein chez BP1818 l'un des architectes m'a permis d'effectuer des tâches sur les environnements de pré-production et de production comme l'installation de la stack ELK, la configuration RabbitMQ ou encore la résolution de certains problèmes d'infrastructure. Cela m'a apporté de nombreuses connaissances, une importante stack et m'a encore une fois permis de sortir de la seule vision "développeur". \\
	
	Fort de cette expérience, j'ai accepté l'offre d'emploi que Sopra Steria m'a proposé, les deux projets ayant comblés mes attentes et m'ayant permis de grandement m'épanouir. Le projet chez Neuflize OBC étant en phase de maintenance et l'application étant en production, je suis maintenant passé à temps plein chez BP1818. Comme nous l'avons déjà dit au début de ce rapport, cette banque est une filiale de Natixis Bank qui, étant satisfait du travail effectué, souhaite voir de nouveaux projets fleurir. Ainsi, Natixis Bank Luxembourg souhaite elle aussi mettre en place une application web très semblable au fronting digital de BP1818, projet qui pourrait être réalisé avec l'équipe actuelle qui connait bien le produit.
	
\end{changemargin}
%-----------------------------------------------------------------------------------------------------------------
%-----------------------------------------------------------------------------------------------------------------


%-----------------------------------------------------------------------------------------------------------------
%-----------------------------------------------------------------------------------------------------------------
% BIBLIOGRAPHIE
\begin{thebibliography}{9}
	\addcontentsline{toc}{chapter}{Bibliographie}	% permet de l'avoir dans le sommaire

%	\bibitem{Livre 01}
%		\textsc{Nom}, Prénom
%		\textit{Titre},
%		edition, date.
%
% USE : \cite{Livre 01}

\end{thebibliography}
%-----------------------------------------------------------------------------------------------------------------
%-----------------------------------------------------------------------------------------------------------------

%-----------------------------------------------------------------------------------------------------------------
%-----------------------------------------------------------------------------------------------------------------
% ANNEXES
\begin{appendices}

	\chapter{A1}
	\label{a1}
	
\end{appendices}
%-----------------------------------------------------------------------------------------------------------------
%-----------------------------------------------------------------------------------------------------------------
\newpage
\newpage
% Page blanche

\end{document}
