\documentclass[11pt,twoside]{scrreprt}
\usepackage[utf8]{inputenc} 	%package pour le français sous ubuntu : à vous d'adapter
\usepackage[french]{babel}	%pour le français
\usepackage[T1]{fontenc}	%pour les polices
\usepackage{amsmath}		%pour des maths
\usepackage{amsfonts}		%pour des maths
\usepackage{amssymb}		%pour des maths
\usepackage{graphicx}		%pour les inclusions de graphiques
\usepackage{times}		%choix personnel de police : Times New Roman 
\usepackage{lscape}		%si vous voulez des images en paysage
\usepackage{hyperref}		%pour les liens croisés à travers le fichier .pdf
\usepackage{fancyhdr}		%pour les marges
\usepackage{float, caption}	%pour le positionnement et légende des images
\usepackage{color}		%pour la couleur dans le code
\usepackage{listings}		%pour mettre du code, plutôt en Annexe
\usepackage{lastpage}		%pour les références aux pages
\usepackage{epic,eepic}		%pour le positionnement de l'image de garde
\usepackage{wrapfig}		%pour les images sur le coté dans le texte
\usepackage{calc,ifthen,xspace}	%pour redéfinir les espaces et distances
\usepackage{scrlfile}
\usepackage{soul} %surlignance
\PreventPackageFromLoading{fp}
\usepackage[final]{pdfpages} % Pour inclure un pdf
\usepackage[titletoc]{appendix} % Pour les annexes
\usepackage{glossaries} % Pour le lexique
\usepackage{tabularx}
\usepackage[figuresright]{rotating}
\usepackage{diagbox}
\ResetPreventPackageFromLoading
\usepackage{array}
\pagestyle{fancy}
\AddThinSpaceBeforeFootnotes 
\FrenchFootnotes 
\makeglossaries
\usepackage{lmodern}
\usepackage{textcomp}
\usepackage{xspace}
\usepackage{listingsutf8}
\usepackage{xcolor}
\usepackage{afterpage}
\usepackage{url}
\usepackage[top=2.1cm,bottom=2.3cm,left=2cm,right=2cm]{geometry}
\usepackage{multirow}
\usepackage{verbatim}

% Pour sommaire cliquable 

\hypersetup{
dvips,
backref=true, %permet d'ajouter des liens dans...
pagebackref=true,%...les bibliographies
hyperindex=true, %ajoute des liens dans les index.
colorlinks=false, %colorise les liens
breaklinks=true, %permet le retour à la ligne dans les liens trop longs
urlcolor= blue, %couleur des hyperliens
linkcolor= blue, %couleur des liens internes
bookmarks=blue, %créé des signets pour Acrobat
bookmarksopen=true} 

\definecolor{hellgelb}{rgb}{1,1,0.8} % couleur pour le code
\definecolor{colKeys}{rgb}{0,0,1}
\definecolor{colIdentifier}{rgb}{0,0,0}
\definecolor{colComments}{rgb}{0,0.5,0}
\definecolor{colString}{rgb}{0.62,0.12,0.94}
\definecolor{INSA_GM}{cmyk}{0.6,0,0,0} % et la page de garde
\definecolor{INSA_GRIS}{cmyk}{0.7,0.6,0.5,0.3}
\definecolor{INSA_BLEU}{cmyk}{1,0.9,0.1,0}
\definecolor{mygreen}{rgb}{0,0.6,0}
\definecolor{mygray}{rgb}{0.5,0.5,0.5}
\definecolor{mymauve}{rgb}{0.58,0,0.82}
\definecolor{mygrayblack}{rgb}{0.32,0.32,0.32}
\colorlet{punct}{red!60!black}
\definecolor{background}{HTML}{EEEEEE}
\definecolor{delim}{RGB}{20,105,176}
\colorlet{numb}{magenta!60!black}
\definecolor{backcolorLogstash}{rgb}{0.95,0.95,0.92}

\lstdefinestyle{logstash}{
	backgroundcolor=\color{backcolorLogstash},
	basicstyle=\ttfamily,
	breaklines=true
}

\lstdefinelanguage{json}{
    basicstyle=\normalfont\ttfamily,
    numbers=left,
    numberstyle=\scriptsize,
    stepnumber=1,
    numbersep=8pt,
    showstringspaces=false,
    breaklines=true,
    frame=lines,
    backgroundcolor=\color{background},
    literate=
     *{0}{{{\color{numb}0}}}{1}
      {1}{{{\color{numb}1}}}{1}
      {2}{{{\color{numb}2}}}{1}
      {3}{{{\color{numb}3}}}{1}
      {4}{{{\color{numb}4}}}{1}
      {5}{{{\color{numb}5}}}{1}
      {6}{{{\color{numb}6}}}{1}
      {7}{{{\color{numb}7}}}{1}
      {8}{{{\color{numb}8}}}{1}
      {9}{{{\color{numb}9}}}{1}
      {:}{{{\color{punct}{:}}}}{1}
      {,}{{{\color{punct}{,}}}}{1}
      {\{}{{{\color{delim}{\{}}}}{1}
      {\}}{{{\color{delim}{\}}}}}{1}
      {[}{{{\color{delim}{[}}}}{1}
      {]}{{{\color{delim}{]}}}}{1}
      {"}{{{\color{mygreen}{"}}}}{1},
}
	
\usepackage{templateINSA}
\initINSA

% Titre centre
\renewcommand\infoBig{Leriche Florian}
\renewcommand\infoSmall{Rapport de stage ingénieur}

% Titre bas 
\title{Le Digital Banking}
\renewcommand\soustitre{Sopra Steria Group}

% Auteurs
\author{ 
	\textbf{Étudiant :} \bsc{Leriche} Florian\\
	\textbf{Maîtres de stage :} \bsc{Berthaud} Laurent, \bsc{Sacenda} Cyril \\
	\textbf{Entreprise :} Sopra Steria\\
	\textbf{Clients :} BP1818, Neuflize OBC\\
	\textbf{Date :} 13/02/2017 - 12/07/2017\\
	\textbf{Lieu :} Paris, France\\
}

\begin{document}
\thispagestyle{empty}
% titleINSA : Page de garde 
% #1 : descendre le titre du milieu (en mm)
% #2 : lien de l'image de fond
% #3 : décalage sur X de l'image de fond (en mm)
% #4 : décalage sur Y de l'image de fond (en mm)
% #5 : largeur de l'image de fond de #5 (en mm)
% #6 : Crédit de l'image de fond
\titleINSA{15}{images/fond.jpg}{-20}{60}{250}{Image : \href{http://img.decision-achats.fr/Img/BREVE/2016/1/300834/Digitalisation-opportunite-achats-porteurs-innovation--F.jpg}{\color{white}{http://img.decision-achats.fr/}}}

\thispagestyle{empty}

%-----------------------------------------------------------------------------------------------------------------
%-----------------------------------------------------------------------------------------------------------------
	\tableofcontents
	\addtocontents{toc}{\protect\thispagestyle{empty}}
	\thispagestyle{empty}
%-----------------------------------------------------------------------------------------------------------------
%-----------------------------------------------------------------------------------------------------------------

\newpage

\chapter*{Remerciements}
	\addcontentsline{toc}{chapter}{Remerciements}
	\setcounter{page}{1}
	\setcounter{tocdepth}{1}
Je tiens en premier lieu à remercier Sopra Steria Group de m’avoir accueilli durant ce stage et de m’avoir donné la possibilité à travers mon stage d’acquérir de l’expérience dans un domaine qui me tient à cœur, de découvrir un environnement de travail en adéquation avec mes études et m’ayant permis de mettre en pratique les connaissances acquises au cours de ma formation théorique. \\

Je remercie, en particulier, mes tuteurs de stage Cyril Sacenda et Berthaud Laurent pour m’avoir encadré et sans qui ce stage n’aurait pas pu avoir lieu. Je les remercie eux ainsi que l'équipe RH pour toutes les démarches qu’ils ont pu entreprendre à mon égard. \\

Je remercie également grandement les développeurs Bouhamyd Mohamed et Azzerrifi Laamrani Abdellatif côté Neuflize OBC et Rinfray Corentin et Madih Jaafar côté BP1818 ainsi que les architectes Rudy Jansem, Barillet Cyril et N'Takpe Jocelyn pour lesquels j’ai travaillé directement, pour m’avoir conseillé et aidé lorsque j’en avais besoin. Je suis extrêmement reconnaissant du temps qu’ils ont pu m’accorder et de l’enrichissement qu’ils ont pu m’apporter en partageant leur connaissance avec moi. \\

Enfin, je remercie tout le personnel avec lequel j’ai eu l’occasion de travailler et qui m’ont beaucoup apporté et aidé lors de mon stage. \\

En outre, je remercie aussi la communauté architecte du marché Banque et finance pour les Brown Bag Lunch (présentation de technologies et retour d'expérience sur l'heure du repas) ainsi que l'équipe commerciale pour m'avoir fait intervenir dans le cadre d'un retour d'expérience à la soirée du Digital Banking.

%-----------------------------------------------------------------------------------------------------------------
%-----------------------------------------------------------------------------------------------------------------
	\chapter*{Introduction}				% garde la même mise en page qu'un chapitre mais ne le numérote pas
	\addcontentsline{toc}{chapter}{Introduction}	% permet de l'avoir dans le sommaire

\paragraph{}
Le présent rapport détail le travail qui s’inscrit dans le cadre de la réalisation d’un stage ingénieur effectué par les élèves ingénieurs de l’Institut National des Sciences Appliquées de Rouen. Celui-ci s’est déroulé via la participation à deux missions distinctes pour le compte de la société de services \textit{Sopra Steria} du 13 février au 11 août 2017; suite à une convention tripartite signée entre le département Architecture des Systèmes d’Information de L’INSA, l’entreprise d’accueil et moi-même. Pendant ce stage, j’ai eu l’occasion de participer à plusieurs projets qui m’ont permis d’appréhender le métier d’ingénieur en informatique, d’acquérir de l’expérience ainsi que de m’épanouir aussi bien dans le plan personnel que professionnel.

\paragraph{}
La première mission à laquelle j'ai pris part s'est déroulée du 13 février 2017 jusqu'à la fin de mon stage, pour le compte du client \textit{Neuflize OBC}. La seconde mission a débuté le 29 mars 2017 pour la banque privée \textit{BP1818} et s'est déroulée en parallèle de la première jusqu'à la fin du stage. La répartition du temps de travail était la suivante :
\begin{itemize}
	\item Lundi, mardi et vendredi chez le client \textit{BP1818}
	\item Mercredi et jeudi chez le client \textit{Neuflize OBC}
\end{itemize}

\paragraph{}
Sopra Steria est une entreprise de services du numérique et l'un des leader européens dans la transformation numérique. Ainsi, l'objectif premier de mon stage a été d'accompagner certains des clients banques privées de Sopra Steria dans leur transformation digital en travaillant à la fois sur la réalisation et l'industrialisation d'une application mobile à destination des clients de la banque Neuflize ainsi que sur la mise en place d'une application web à destination des banquiers travaillant chez BP1818.

\paragraph{}
Dans un premier temps, je présenterai l'entreprise d'accueil et les client pour lesquels j'ai travaillé, leurs domaines d’activités, leurs origines, leurs personnel ainsi que leurs locaux.
Dans un second temps, je détaillerai de manière précise les sujets des missions qui m’ont été confiées au sein des équipes.
Enfin, je décrirai en profondeur le déroulement de mon stage ainsi que les différents travaux que j’ai accompli et les conditions dans lesquelles je les ai réalisés. 

\paragraph{}
Pour des raisons de clarté et de cohérence, ce rapport présentera deux parties distinctes concernant le déroulement du stage, chacune ayant pour objectif de décrire le travail réalisé chez les clients cités plus haut et suivant le même schéma.
%-----------------------------------------------------------------------------------------------------------------
%-----------------------------------------------------------------------------------------------------------------

%-----------------------------------------------------------------------------------------------------------------
%-----------------------------------------------------------------------------------------------------------------	
\chapter{Présentation de l'entreprise}

	\section{Historique}
	\paragraph{}
\textit{Sopra Steria} est une entreprise de services du numérique (ESN) résultant de la fusion, en janvier 2015, de deux des plus anciennes sociétés de services en ingénierie informatique françaises, \textit{Sopra} et \textit{Steria}.

\subsection{1968 - 1985 : Les débuts}

\paragraph{}
Création des sociétés Sopra et Steria respectivement en 1968 et 1969, période marquée par la naissance de l'industrie des services informatiques.

La \textbf{SO}ciété de \textbf{PR}ogrammation et d'\textbf{A}nalyses (Sopra), fondée par Pierre Pasquier et François Odin, est avant tout une entreprise de services de conseils technologiques spécialisée dans l'édition de logiciel. Elle signera, quelques années plus tard, son premier grand contrat d'infogérance bancaire qui marquera son initiation au savoir-faire relatif à la banque. Cela aboutira à la création de la première plateforme bancaire de Sopra qui proposera par la suite des logiciels à destination des banques. Par la suite, l'édition de solutions bancaires deviendra son activité phare avec la mise en production de sa première application concernant les crédits.

La \textbf{S}ocié\textbf{T}é d'\textbf{E}tude et de \textbf{R}éalisation en \textbf{I}nformatique et \textbf{A}utomatisme (Steria) contrôlée par Jean Carteron est aussi une société de services informatiques. L'informatisation de l'Agence France Presse est désignée comme étant l'une des première prouesse de la société qui participera par la suite au développement du minitel en travaillant sur la conception de son architecture.

\subsection{1985 - 2000 : Croissance et entrée en bourse}

\paragraph{}
Sopra repense sa structure industrielle en décidant de se recentrer sur des activités précises telles que l'intégration de systèmes et l'édition de logiciels et décroche son premier grand projet national avec le ministère de l'intérieur. Elle est introduite en Bourse en 1990 et multiplie les contrats ainsi que les acquisitions avec, par exemple, le rachat de \textit{SG2 ingénierie} marquant une forte croissance. Elle profitera de ses performances pour étendre son expertise à l'échelle internationale en s'implantant dans différents pays tels que le Royaume-Uni ou encore l'Allemagne.\\

De même, Steria étend son influence en dehors de la France, jusqu'en Arabie Saoudite où elle réalisera le système informatique de la banque centrale saoudienne. Elle intégrera le marché des transports à son domaine d'expertise, notamment grâce au projet d'automatisation du RER A en France, à Paris. Ses futures acquisitions lui permettront une entrée en Bourse en 1999.\\

\subsection{2000 - 2014 : Transformation numérique}

\paragraph{}
L'essor des technologies du numériques, à savoir l'informatique et internet, provoque une mutation des marchés qui a pour conséquence d'apporter de nombreux clients potentiels à la recherche de partenaires pouvant les accompagner dans leur transformation numérique. Les deux sociétés répondront à cette problématique et développeront leurs activités de conseil. Sopra crééra sa filiale \textit{Axway} regroupant ses activités dans le domaine du progiciel et qui entrera aussi en Bourse de manière autonome en 2011. Toujours fortement impliquée dans le domaine bancaire, elle décidera de créer sa filiale \textit{Sopra Banking Software} en 2012 et réalisera de nombreux projets avec les grands noms du secteur bancaire français (Crédit agricole, Société général, BNP, etc...)\\

Steria, de son côté, se retrouvera dans un contexte économique difficile et verra le prix de son action chuter. Elle continuera malgré tout de multiplier les acquisitions (\textit{Bull} en Europe, \textit{Mummert Consulting} en Allemagne ou encore \textit{Xansa} au Royaume-Uni). Elle remportera plusieurs des plus gros contrats (notamment SSCL pour la gestion du back office de plusieurs ministères de l'administration britannique) de son histoire avec le gouvernement britannique.

\subsection{2014 - 2016 : Fusion absorption}

\paragraph{}

En 2014, Sopra, forte d'une grande croissance, prend la décision d'absorber Steria en devenant actionnaire majoritaire à plus de 90\%. Il s'agit là d'une excellente opportunité misant sur la complémentarité des deux géants de l'informatique aussi bien sur le plan métier que sur le plan géographique. En effet, comme nous l'avons dit plus haut, Steria est très présente au Royaume-Uni contrairement à Sopra. De plus, Sopra se verra ainsi faire l'acquisition de nombreux centres de compétences qui viendront renforcer son écrasante influence à travers l'Europe. Les deux entreprises partagent beaucoup de points communs tels que la taille, les domaines d'activités ou encore les objectifs, ce qui constitue un atout majeur concernant leur fusion et leur projets d'avenir. La fusion des deux groupes donne donc naissance à Sopra Steria, une entreprise possédant un poids écrasant, un capital titanesque permettant de gagner la confiance de nombreux clients ainsi qu'une très grande expertise lui permettant de mener à bien beaucoup de projets emblématiques tels que : \\

\begin{figure}[h]
	\includegraphics[scale=0.8]{images/entreprise/projetsEmblematiques.png}
	\centering
	\caption{Projets emblématiques}
	\label{projetsEmblematiques}
\end{figure}
		

\section{Domaine d'activités}
	\paragraph{}

\textit{Sopra Steria} est une entreprise de services du numérique (ESN) résultant de la fusion, en janvier 2015, de deux entreprises françaises, \textit{Sopra} et \textit{Steria} créées respectivement en 1968 et 1969. Experte dans le domaine de l'inforamtique et leader européen de la transformation numérique, elle offre une très grande variété de service à ses clients tel que le conseil, l'intégration de systèmes, la cyber sécurité, l'éditions de solutions métiers, la gestion d'infrastructure ou encore le business process services.  

\section{Quelques chiffres}
	\begin{figure}[h]
	\includegraphics[scale=1]{images/sopraSteriaChiffres.png}
	\centering
	\caption{Quelques chiffres}
	\label{sopraSteriaChiffres}
\end{figure}

\begin{figure}[h]
	\includegraphics[scale=0.9]{images/sopraSteriaMonde.png}
	\centering
	\caption{Répartition internationale}
	\label{sopraSteriaMonde}
\end{figure}
		
		
	
\section{Organisation}
	\begin{figure}[h]
	\includegraphics[scale=0.8]{images/sopraSteriaOrganisation.png}
	\centering
	\caption{Organisation du groupe}
	\label{sopraSteriaOrganisation}
\end{figure}
		
		
\section{Partenaires et clients}
	\subsection{Neuflize OBC}

\begin{figure}[h]
	\includegraphics[scale=0.3]{images/neuflizeOBCLogo.png}
	\centering
%	\caption{Logo Neuflize OBC}
	\label{neuflizeOBCLogo}
\end{figure}

\subsection{BP1818}

\begin{figure}[h]
	\includegraphics[scale=0.3]{images/bp1818Logo.png}
	\centering
%	\caption{Logo BP1818}
	\label{bp1818Logo}
\end{figure}
		
		
%-----------------------------------------------------------------------------------------------------------------
%-----------------------------------------------------------------------------------------------------------------

%-----------------------------------------------------------------------------------------------------------------
%-----------------------------------------------------------------------------------------------------------------

\chapter{Présentation des sujets}
	
		Le secteur bancaire est actuellement en effervescence, impacté par la venue de nombreux nouveaux acteurs qui viennent bouleverser les règles établies. En effet, le progrès dans le domaine de l'informatique a permis de fournir des services inédits pouvant répondre aux attentes de la nouvelle génération de clients marquée par l'essor des technologies du numérique. Ainsi, de nouveaux acteurs sont apparus : les banques en ligne, c'est-à-dire des banques uniquement disponibles sur internet. Ces dernières ne possèdent pas de locaux physiques et n'ont que très peu de personnel. Néanmoins, elles sont capables de permettre à leurs clients de suivre l'état de leur compte bancaire ou patrimoine financier en temps réel. Il est possible de commander une carte bleue, un chéquier ou encore d'obtenir un rib sans avoir à se déplacer et bien d'autres services sont disponibles selon les banques et cela gratuitement. Les coûts de gestion des ressources humaines, d'organisation ou encore de matériels sont grandement réduits ce qui permet à ces nouvelles banques de pouvoir défier les grands groupes présents à l'échelle internationale. Ces derniers ont donc la nécessité de réagir afin de rester compétitifs sur ce marché en pleine évolution. \\

	On parle ainsi de transformation digitale des banques pour désigner la transition marquée par le processus de dématérialisation de l'économie en faisant appel aux technologies modernes. Neuflize OBC, comme nous l'avons vu dans la partie précédente, est une banque privée. Cette dernière est actuellement en cours de transformation afin de pouvoir répondre aux besoins de ses clients. Un \href{https://www.neuflizeobc.net/portail/portail.jsp}{site web} permettant de proposer les services dits "banque au quotidien" qui regroupe les fonctionnalités classiques à savoir consultation de comptes, impression de rib ou encore réalisation d'une transaction bancaire existe depuis plusieurs années. Cependant, les exigences sont toujours plus élevées, la génération actuelle étant toujours connectée via l'utilisation d'un smartphone, Neuflize a le besoin de produire une application mobile dans le but de permettre à ses clients de pouvoir accéder à leurs informations n'importe où et n'importe quand. \\

	L'équipe de développement constituée par Sopra Steria était en charge de la réalisation de la partie \textit{backend} de l'application, la partie \textit{frontend} ayant été déléguée à l'équipe qui a conçu le site web. Neuflize a décidé d'exposer des API dans le but de permettre la réalisation d'échanges au sein de son SI et vers l'extérieur. Ainsi, une architecture multicouches a été mise en place avec notamment :

\begin{itemize}
	\item Une couche d'API Management
	\item Une couche de microservices
	\item Une couche API Backend \\
\end{itemize} 

	La couche backend a pour objectif d'exposer des services unitaires développés par \textit{Elcimaï Financial Software} (EFS), un éditeur spécialisé dans la dématérialisation des flux financiers, créateur de la solution \textbf{WeBank}. Cette solution logicielle propose des services grandement sécurisés permettant de mettre en place de l'authentification via l'utilisation de tokens, de la gestion de portefeuilles titres ou encore la signature numérique de transactions bancaires. Ces web services sont actuellement utilisés par le site internet de la banque et sont donc réemployés pour l'application mobile afin d'assurer une certaine cohérence. \\

	La couche microservices est au coeur du sujet de ce stage. En effet, j'ai intégré l'équipe de projet en charge du développement de cette couche. Cependant, la phase de développement touchant à sa fin, j'ai principalement participé à la phase d'industrialisation via la réalisation de certains travaux indépendants (automatisation de tests fonctionnels, tests de charges ou encore dashboards pour le client). C'est en majeure partie pour cette raison que j'ai été assigné sur un second projet dont le développement venait tout récemment de commencer.
%-----------------------------------------------------------------------------------------------------------------
%-----------------------------------------------------------------------------------------------------------------
	
%-----------------------------------------------------------------------------------------------------------------
%-----------------------------------------------------------------------------------------------------------------	
\chapter{Neuflize OBC}

	\fancyhead[LE]{
	\begin{picture}(0,0) 
	\put(-30,-8){\includegraphics[width=49mm]{images/neuflizeOBCLogo.png}}
	\end{picture}
}
\fancyhead[LO]{
	\begin{picture}(0,0) 
	\put(-8,-8){\includegraphics[width=49mm]{images/neuflizeOBCLogo.png}}
	\end{picture}
}

\section{Présentation du projet}
\label{prezAppNeuflize}
	\begin{figure}[H]
\raggedleft
	\includegraphics[scale=0.5]{images/travailNeuflizeOBC/architecture/pbiEfs.png}
	\centering
	\caption{Composition des services EFS}
	\label{coucheMicroservices}
\end{figure}

L'API microservices utilise la stack Netflix OSS qui permet d'intégrer les patterns classiques aux application distribuées. Les composants Netflix sont intégrés via Spring Cloud. Les principales briques techniques de l'API sont les suivantes :\\

\begin{itemize}
	\item Un serveur de configuration basé sur Archaius
	\item Un serveur d'annuaire basé sur Eureka
	\item Une gateway implémentée via Zuul
	\item Trois services métiers :
		\begin{itemize}
			\item account-service : gérant les informations liées aux comptes des utilisateurs
			\item profile-service : gérant les informations liées aux profils des utilisateurs
			\item transaction-service : gérant les informations liées aux transactions bancaires
		\end{itemize}
	\item Une interface d'analyses des logs pour le monitoring basée sur la stack ELK : ElasticSearch, Logstash et kibana
	\item Un dashboard basé sur Zipkin
\end{itemize}
	
\section{Architecture du projet mobile}
	\paragraph{}
Dans cette partie nous allons présenter plus en détails l'architecture globale du projet d'application mobile de Neuflize OBC. Comme nous l'avons vu précédemment, ce projet est basé sur une architecture multicouches dont la structure est représentée dans sa globalité figure \ref{archiLog}. Nous allons maintenant décrire l'objectif et le foncitonnement de chacune des couches afin de comprendre le fonctionnement du projet.

\subsection{API Security Gateway}
	TODO : blabla sur la couche
	
\begin{figure}[H]
	\includegraphics[scale=0.5]{images/travailNeuflizeOBC/architecture/coucheSecurity.png}
	\centering
	\caption{Couche API Security Gateway}
	\label{coucheSecurity}
\end{figure}

	\subsubsection{B1 - Node Manager}
	Gestion de la configuration logique (topologie, domaine, instance …), interagit avec la brique B18 (Admin Node manager) afin de permettre la scalabilité horizontale de la solution.
		
	\subsubsection{B2 - API Gateway}
	Serveur traitant les appels API. Cette brique est en charge de la sécurité applicative des appels vers la couche API Management. Positionnée dans le tiers 1, elle recevra les appels « HTTPS » et aura principalement pour objectif d’effectuer les actions suivantes : \\
	
	\begin{itemize}
		\item Terminaison TLS : vérification de la validité certificat partenaire
		\item Filtrage des requêtes 
		\item Firewall applicatif : vérification du contenu des messages REST
		\item Répartition de charge vers les composants en Aval
		\item Protection du SI : limitations du nombre d’appels API (mécanisme de régulation du traffic) et des appels au SI (mécanisme de cache)
		\item SLA (service Level Agreement) : collecte et trace des exécutions \\
	\end{itemize}
	
	\subsubsection{B3 - EhCache}
	Système de gestion de cache distribué en mémoire.
	
	\subsection{API Management}
	
	TODO : blabla sur la couche
	
\begin{figure}[H]
	\includegraphics[scale=0.5]{images/travailNeuflizeOBC/architecture/coucheManagement.png}
	\centering
	\caption{Couche management}
	\label{coucheManagement}
\end{figure}

	\subsubsection{B4 - API Manager}
	Brique permettant de configurer et d’exposer des API. Elle contient également un mini serveur http pour proposer des pages statiques d’authentification utilisateurs. Elle assurera les fonctionnalités suivantes : \\
	
	\begin{itemize}
		\item Publication et sécurisation des API
		\item Gestion du cycle de vie des API
		\item Gestion de l’authentification et des habilitations (développeurs et administrateurs API)
		\item Embarquement des développeurs d’applications consommatrices d’API
		\item Audit, suivi de la consommation des API, gestions des quotas
		\item Haute disponibilité \\
	\end{itemize}

Deux instances d’API Gateway seront installées en mode « actif/actif ». La répartition de charge sera gérée par le composant API Gateway positionné en amont.
	
\subsection{Microservices}

	L'objectif de cette couche est de réaliser la composition des services métiers exposés par EFS dans le but d'exposer les données pour les applications ou les partenaires. L'architecture microservices est un paradigme d'architecture qui jouit actuellement d'une grande popularité aux dépends de celles plus classiques (N-tiers, SOA...), inventée afin de répondre aux problématiques soulevées par les projets de grande ampleur. \\
	
	Cette approche consiste à développer une application sous forme d'un ensemble de services dont la granularité correspond à une fonctionnalité élémentaire en terme métier. Chacun de ces services doit posséder son propre contexte d'exécution et ainsi être testable et déployable indépendemment en favorisant un couplage le plus faible possible. Ils peuvent être écrit dans des langages différents et communiquer entre eux via, par exemple, le protocole HTTP et la mise en place d'une API REST, ce qui est le cas pour ce projet. On parle alors de microservices, terme qui s'oppose aux applications plus classique que l'on dit monolithiques.\\
	
	Dans les gros projets, la quantité de code a tendance à augmenter rapidement impliquant une hausse de la compléxité et rendant ainsi difficile l'ajout de nouvelles fonctionnalités. Le couplage entre ces dernières devient fort et les nombreux effets de bords résultant de chaque modifications rendent alors l'application moins fiable, limitant les perspectives d'évolution. De plus, la scalabilité horizontale (par exemple un ajout de serveur) est elle aussi impactée. En effet, l'application entière doit être migrée si l'on souhaite changer de matériels afin d'améliorer les performances. Si un certain module est plus lent, il n'est pas possible de le déplacer indépendemment afin d'améliorer son exécution, il faut répliquer le monolithe entier.
	
	TODO : finir le blabla sur la couche
	
\begin{figure}[H]
	\includegraphics[scale=0.5]{images/travailNeuflizeOBC/architecture/coucheMicroservices.png}
	\centering
	\caption{Couche microservices}
	\label{coucheMicroservices}
\end{figure}

	\subsubsection{B5 - Spring Boot Stack}
	
Brique applicative hébergeant la couche micro service. C’est dans cette couche que de la composition de services pourra être réalisée (ex : services de l’API Backend EFS). Elle prendra en charge l’implémentation des principaux patterns d’implémentation de cette couche à savoir :
\begin{itemize}
	\item[Circuit breaker] : Capacité du système à être tolérant à la panne. En cas d’erreur successive lors de l’appel d’un sous-composant, le circuit d’appel est coupé « temporairement » en adoptant un comportement par défaut. 
	\item[Feature toggle] : Le principe est d’avoir une branche de développement et de déployer en production en continu. Ensuite l’activation d’une « feature » est pilotée par le business. Cela permet aussi d’activer une fonctionnalité en fonction d’une population ou une stratégie particulière. \\
\end{itemize}
	
	\subsubsection{B6 - Zuul}
	Gateway de l’architecture microservices fournissant des services de routage dynamique, surveillance, résilience et sécurité. Il sera notamment le point d’entrée unique de toutes les requêtes vers la couche micro service. L’implémentation choisie est Zuul de Netflix.
	
	\subsubsection{B7 - Eureka}
	Eureka est le serveur d’annuaire de services. Brique essentielle d'une architecture distribuée, le serveur d'annuaire permet la détection automatique des instances déployées. Les instances des applications sont accédées via leur nom plutôt que par leurs adresses physiques/IPs. Les applications n'ont plus besoin de connaitre les adresses des instances.
	
	\subsubsection{B8 - Hystrix}
	Hystrix est l'implémentation du pattern Circuit breaker permettant de contrôler la latence et les erreurs dues à des appels réseaux. L'idée essentielle est d'empêcher les erreurs en cascade dans un environnement distribué. Hystrix permet de 'fail-fast' mais de se rétablir rapidement créant ainsi une architecture tolérante aux erreurs capable de se rétablir de manière autonome (self-heal). Hystrix encapsule les appels extérieurs dans un thread à part permettant de configurer une méthode de fallback en cas d'erreur. Dès sa conception, le système prévoit les pannes.
De plus, Hystrix remonte des indicateurs concernant le résultat de la requête et le temps de réponse.

	\subsubsection{B9 - Ribbon}
	Librairie RPC gérant la communication inter-processus et qui fournit notamment des fonctionnalités de load-balancing côté client.
	
	\subsubsection{B10 - Reign}
	Librairie facilitant la création de service REST de façon déclarative.
	
	\subsubsection{B11 - Archaius}
	Archaius est le serveur de configuration. Il permet d’avoir une configuration centralisée pour les systèmes distribués. La configuration sera chargée directement depuis le repository de source GIT (brique B23).
	
	\subsubsection{B12 - RxJava}
	RxJava est une implémentation Java de Reactive Extensions : une bibliothèque permettant de créer des programmes asynchrones et événementiels en utilisant des séquences observables.
Il étend le modèle d'observateur pour prendre en charge les séquences de données / événements et ajoute des opérateurs qui permettent de composer des séquences de façon déclarative tout en s’abstrayant des problématiques telles que le bas-niveau threading, la synchronisation, la sécurité des threads et les structures de données concurrentes.
	
	\subsubsection{B13 - Redis}
	Redis cache manager blabla.

\subsection{Backend}

	TODO : blabla sur la couche
	
\begin{figure}[H]
	\includegraphics[scale=0.5]{images/travailNeuflizeOBC/architecture/coucheBackend.png}
	\centering
	\caption{Couche backend}
	\label{coucheBackend}
\end{figure}

	\subsubsection{B14 - EFS WeBank}
	Dans cette partie nous allons présenter plus en détails l'architecture globale du projet d'application mobile de Neuflize OBC. Comme nous l'avons dans la partie précédente, ce projet est basé sur une architecture multicouches dont la composition est représentée figure

\subsection{KPS Data}

	TODO : blabla sur la couche
	
\begin{figure}[H]
	\includegraphics[scale=0.5]{images/travailNeuflizeOBC/architecture/coucheCassandra.png}
	\centering
	\caption{Couche KPS - Apache Cassandra}
	\label{coucheCassandra}
\end{figure}
	
	\subsubsection{B15 - Apache Cassandra}
	Apache Cassandra est un système de gestion de base de données (SGBD) de type NoSQL conçu pour gérer des quantités massives de données sur un grand nombre de serveurs, assurant une haute disponibilité en éliminant les points individuels de défaillance. Il permet une répartition robuste sur plusieurs centres de données3, avec une réplication asynchrone sans master et une faible latence pour les opérations de tous les clients. 
Apache Cassandra est requise pour stocker les données utilisées par le composant API Manager, par exemple le catalogue API, les quotas d’utilisation des API, les clients des API, etc… Cassandra peut être aussi utilisée pour le stockage des composants API Gateway suivants : \\

\begin{itemize}
	\item Key Property Store : table utilisé par API Gateway Server pour conserver des données utilisées lors de l’exécution des requêtes
	\item Magasin de jetons Oauth
	\item Répertoire Client : API Key et les données Oauth utilisées pour la sécurisation des API \\
\end{itemize}

\begin{figure}[H]
	\includegraphics[scale=0.5]{images/travailNeuflizeOBC/architecture/coucheDashboard.png}
	\centering
	\caption{Couche KPS - Dashboards}
	\label{coucheDashboard}
\end{figure}
	
	\subsubsection{B16 - Admin Node Manager}
	C’est le serveur d’administration central d’un domaine API Gateway. Il permet notamment de réaliser toutes les opérations de déploiement, gestion de configurations dynamiques et surveillance de l’activité des instances API Gateway.
	
	\subsubsection{B17 - Kibana}
	Kibana est le module de Dashboard d'ElasticSearch. Il permet d'associer la puissance du moteur de recherche d'ElasticSearch (des recherches complexes peuvent être faites pour filtrer les données pertinentes à l'analyse) aux modules de reporting classiques.
	
	\subsubsection{B18 - Hystrix Dashboard}
	Permet d’effectuer du monitoring et de la gestion d’erreurs sur les services et applications grâce à un tableau de bord présentant les graphes et les métriques sur l’état des services de la plateforme.

\subsection{Monitoring}

	TODO : blabla sur la couche
	
\begin{figure}[H]
	\includegraphics[scale=0.5]{images/travailNeuflizeOBC/architecture/coucheMonitoring.png}
	\centering
	\caption{Couche monitoring}
	\label{coucheMonitoring}
\end{figure}

	\subsubsection{B19 - ElasticSearch}
	Elasticsearch est un serveur utilisant Lucene (une bibliothèque open source écrite en Java qui permet d'indexer et de chercher du texte) pour l'indexation et la recherche des données. Il fournit un moteur de recherche distribué et multi-entité à travers une interface REST. C'est un logiciel libre écrit en Java et publié en open source sous licence Apache.
L'indexation des données s'effectue à partir d'une requête HTTP PUT. La recherche des données s'effectue avec la requête HTTP GET. Les données échangées sont au format JSON.
	
	\subsubsection{B20 - Logstash}
	Logstash est un outil pour collecter, traiter et transférer des événements et des messages de journal. La collecte s'effectue via des plugins d'entrée configurables, y compris la communication socket / paquet brute, le transfert de fichiers et plusieurs clients de bus de messages. Une fois qu'un plugin d'entrée a collecté des données, il peut être traité par un nombre quelconque de filtres qui modifient et annotent les données d'événement. Enfin Logstash envoie des événements aux plugins de sortie qui peuvent transmettre les événements à une variété de programmes externes y compris Elasticsearch, des fichiers locaux et plusieurs implémentations de bus de message.
	
\begin{figure}[h]
	\includegraphics[scale=0.5]{images/travailNeuflizeOBC/architecture/architectureLogicielle.png}
	\centering
	\caption{Architecture logicielle}
	\label{archiLog}
\end{figure}

\newpage
	
\section{Premiers travaux}
	\input{src/travailNeuflizeOBC/premiersTravaux}
	
\section{Tests fonctionnels}
	\subsection{Réalisation des tests fonctionnels}
		Comme nous l'avons vu précédemment, les services de nos API rest sont destinés à être consommés par PBI, en charge du développement de la partie frontend de l'application mobile. Ainsi, nous étions souvent en contact avec ces derniers afin de prendre connaissance des différentes anomalies liées à nos services et de pouvoir répondre à l'évolution de leur besoins. Après correction de celles-ci, il était fréquent que nous soyions amenés à livrer la nouvelle version des services. Ces livraisons permettaient à PBI de pouvoir continuer le développement de l'application dans les meilleures conditions possibles. Elles s'effectuaient sur un serveur d'homologation nommé \textit{homo3} et sur un serveur de pré-production nommé \textit{rgb}. \\
	
	TODO : explication homo3 et rgb\\
	
	Cependant, avant de procéder à la livraison des services, il est nécessaire d'effectuer une batterie de tests fonctionnels permettant de vérifier que le comportement des API est conforme aux spécifications. Ces tests sont essentiels à la satisfaction client et permettent de gagner un temps précieux en évitant d'attendre les retours avant d'identifier les possibles anomalies. Néanmoins, dans notre cas, peu de ces tests avaient été mis en place et il n'existait aucune procédure à suivre. Les cas de tests était rédiger sous Word et le résultat de leur exécution était consigné dans des fichiers excel, peu lisibles, contenant peu d'informations et difficilement traçables. De cette manière, il est difficile de normaliser l'écriture des tests et la création d'un rapport est chronophage (créer les formules excel, etc...) pour un résultat qui ne sera pas exhaustif. En outre, il est impossible d'avoir une vue d'ensemble sur l'évolution des tests au cours du temps et est compliqué de suivre l'état d'une spécification particulière à différentes dates données. Par ailleurs, le client lui-même a souhaité un autre format plus rigoureux permettant de vérifier le comportement des services dans leur intégralités et de pallier aux inconvénients que nous avons cité. \\
	
	Ainsi, j'ai été chargé de mettre en place une procédure pouvant répondre à ce besoin. Cette dernière devait :
	\begin{itemize}
		\item définir les cas de tests
		\item fournir des rapports d'exécution de tests détaillés
		\item nécessiter peu de développement, il était en effet initule de réinventer la roue
		\item être gratuite
		\item être accessible à tout moment à n'importe quel membre de l'équipe de développement qui pourrait être amené à effectuer une livraison
	\end{itemize}
	
	

\subsection{Automatisation des tests fonctionnels}
		Dans le but de répondre aux besoins explicités précédemment, j'ai décidé d'avoir recours à un gestionnaire de tests sous forme d'une application web. En effet, celle-ci pourrait être mise en place sur un seveur interne et rendue disponible pour toute l'équipe, favorisant ainsi le partage d'information et permettant de centraliser toutes les données concernant la réalisation des tests fonctionnels, assurant ainsi un versionning cohérent. De plus, cela répond à la problématique de mise en place rapide et de maintenabilité efficace (pas de mise à jour à gérer sur tous les postes, etc...). \\
	
	Après avoir mené plusieurs recherches, j'ai pu constater qu'il existait différents gestionnaires open-source concurrents sur le marché répondant à nos exigences : \textit{Salomé TMF}, \textit{TestLink} ou encore \textit{Squash TM}. En effet, ces derniers nous offrent tous la possibilité de créer des tests, de les lier aux exigences client ou encore de créer des campagnes de tests puis d'exporter les résultats sous forme de rapport détaillé. Ces outils proposant des fonctionnalités très proches, j'ai décidé d'en choisir un disposant d'une grande flexibilité ainsi que d'une communauté active afin de pouvoir faciliter l'adaptation à nos cas de tests. En effet, les gestionnaires génèrent des rapports regroupant des informations telles que la description des tests, leur temps d'exécution ou leur statuts. Cependant, dans notre cas, nous souhaitions pouvoir inclure pour chacun des tests des informations supplémentaires telles que le nom du service auquel appartient la fonctionnalité testée, l'identifiant de l'utilisateur utilisé pour le test, le numéro de compte bancaire utilisé et de manière générale, tous les paramètres utilisés pour réaliser les requêtes testées. PBI partageant les mêmes données que nous sur les environnements d'homologation et de pré-production, il leur était alors possible de vérifier de leur côté que les tests passent effectivement. De plus, lorsqu'ils remarquaient une anomalie, ils pouvaient nous transmettre les paramètres qu'ils avaient utilisé afin que nous puissions reproduire celle-ci chez nous. \\
	
	Ainsi, j'ai décidé d'utiliser l'outil \textit{TestLink} \cite{bib_testlink} dont la communauté avait mis à disposition de tous des templates permettant de modifier le code source afin de customiser la génération des rapports de tests. Celui-ci est une application web développée en PHP et utilisant le système de gestion de base de données MySQL. Il permet de centraliser toute la gestion des tests fonctionnels du projet en les organisant par le biais des structures présentées dans le tableau \ref{structuresTestlink}.
	
	\newpage

\begin{table}[h!]
	\center
	\begin{tabular}{| c | c |}
     \hline
     Cas de test & Test fonctionnel définissant un scénario spécifique \\ \hline
     Suite de tests & Collection de cas de test validant une même fonctionnalité \\ \hline
     Plan de tests & Collection de suite de tests contenant toutes les informations \\  & telles que la portée, les étapes, la version etc...\\ & Un plan est exécuté pour un build particulier \\ \hline
     Build & Une release spécifique des APIs testées \\
     \hline
	\end{tabular}
	\caption{Structures fournies par TestLink}
	\label{structuresTestlink}
\end{table}

Cet outil présente de nombreux avantages qui m'ont conforté dans mon choix :
\begin{itemize}
	\item Campagnes de tests versionnées dont l'historique est enregistré en base de données.
	\item Export et import de cas de test et de leur résultats
	\item Connection avec \textit{Mantis}, un tracker de bug utilisé dans notre projet
	\item Gestion de rôles sur les tests (qui effectues le test, qui valide etc...)
	\item Rapport complet dans différents formats
	\item Accessible à toute l'équipe n'importe quand 
	\item Simplicité d'utilisation et de mise en place \\
\end{itemize}

	Ensuite, avant de procéder à la création des cas de test sur TestLink, j'ai commencé par définir la structure du futur plan de test qui serait exécuté avant chaque livraison. Ainsi, j'ai décidé de séparer l'ensemble des tests en deux grandes familles : ceux concernant les fonctionnalités de consultation et ceux concernant les fonctionnalités de transaction, ce qui m'amena à la création de deux suites de tests. Après cela, j'ai créé autant de suites de tests qu'il y avait de fonctionnalités décrites dans l'annexe \ref{a2}. Il est possible d'observer sur la figure \ref{testlink} l'organisation d'un plan de test type. Afin de garder le même formalisme tout le long de la réalisation des plans de tests et pour assurer une certaine cohérence, j'ai décidé de mettre en place plusieurs conventions définissant une stratégie de test :
	
	\subsubsection*{Cas de test}
	Les cas de tests doivent avoir un nom de la forme [id]-[titre] où
	\begin{itemize}
		\item id désigne un ID unique permettant de les identifier rapidement et de faciliter leur organisation. Celui-ci est \textit{NOBC-API-XX}, où XX représente le numéro du test. 
		\item titre désigne de manière clair et concise l'objectif du test
	\end{itemize}		
	De plus, chaque cas de test possède en attribut un numéro de version de la forme \textit{vX} où X est incrémenter de 1 chaque fois que le cas de test est modifié.
	
	\subsubsection*{Plan de test}
	Les plans de tests doivent avoir un nom de la forme [scope]-[environnement]-[version] où
	\begin{itemize}
		\item scope désigne la portée du plan de test : "complete" pour tous les tests, "transaction service" pour les tests du service de transaction, "transaction overview" pour les tests de la fonctionnalité transaction overview etc...
		\item environnement désigne le serveur sur lequel sont effectués les tests : homo3 ou rgb
		\item version est de la forme vX.Y.Z où
			\begin{itemize}
				\item X est incrémenté de 1 lorsqu'un nouveau cas de test est ajouté ou supprimé du plan
				\item Y est incrémenté de 1 lorsqu'un cas de test existant du plan a été modifié
				\item Z est incrémenté de 1 à chaque exécution du plan
			\end{itemize}
	\end{itemize}
	
	\subsubsection*{Build}
	Les builds doivent avoir un nom de la forme : [version] où
	\begin{itemize}
		\item version désigne la version de l'API microservices testée \\
	\end{itemize}

	La stratégie de test étant définie, il fallait maintenant déterminer quelles informations devaient être transmises au sein des rapports de tests. Les gestionnaires de tests proposent de remplir des formulaires afin consigner le résultat des tests une fois ceux-ci effectués, ce qui servira par la suite à générer un rapport. Cependant, ces derniers sont plutôt génériques et ne permettent pas de renseigner des données spécifiques à un projet particulier. Comme nous l'avons dit plus haut, nous souhaitions être en mesure de passer des paramètres supplémentaires afin de faciliter nos échanges avec PBI, comme : \\
	
	\begin{itemize}
		\item url et paramètres utilisés pour la requête
		\item service testé
		\item identifiants utilisés
	\end{itemize}
	
	\begin{figure}[h!]
		\includegraphics[scale=0.5]{images/travailNeuflizeOBC/testsFonc/testlink.png}
		\centering
		\caption{Configuration d'un plan de test type}
		\label{testlink}
	\end{figure}
	
	Ainsi, j'ai décidé de modifier le code source de TestLink afin de rajouter les champs dont nous avions besoin aux formulaires de tests. J'ai réalisé cette modification en deux étapes dont la première consistait à ajouter les champs aux formulaires puis à gérer la partie front en PHP. La seconde concernait la récupération des données et leur sauvegarde en base de données, ce qui a impliqué la création de nouvelles requêtes SQL. Une fois les champs mis en place, j'ai créé un cas de test puis générer un premier rapport au format PDF en guide de POC (proof of concept). Celui-ci ayant été jugé satisfaisant, j'ai dû étudier l'ensemble des spécifications du projet concernant chacun des services mis en place afin de procéder à l'écriture de tous les cas de tests. Cela m'a permis de mieux comprendre le besoin du client, d'avoir une bien meilleure vision sur le projet dans sa globalité en m'apportant des informations sur l'utilité de chaque service et de pouvoir me former sur le projet en restant productif.
	
	Ces cas de tests permettaient de vérifier que les réponses des requêtes émises vers l'API microservices étaient en accord avec les attentes de PBI. Par exemple, les réponses étant au format JSON, ils permettaient la vérification de la présence de tous les champs obligatoires, la cohérence des valeurs des champs (par exemple un compte de type emprunt aura un champ "montant emprunté" alors qu'un compte épargne n'en aura pas) ou encore la vérification de la cohérence des données de notre backend avec celles de l'application web.
	
\subsection{Réalisation des tests de charge}
		Une fois tous les cas de tests rédigés, j'ai procédé à la réalisation d'une campagne de tests complète aboutissant à la génération d'un rapport. Afin de procéder à cela, pour chacun des tests j'ai utilisé l'outil \textit{Postman} qui est une plateforme proposant une interface graphique facilitant la construction de requêtes. Cet outil est conçu pour faciliter le développement des APIs en permettant de les intéroger de manière très rapide et simple sans avoir à développer un client pour les consommer. Dans l'optique de mettre à profit les tests effectués, nous avons connecté TestLink à Mantis, une application web permettant d'assurer le suivi des anomalies dans laquelle il est possible d'ouvrir des tickets concernant des bugs qui seront alors pris en charge par les développeurs jusqu'à leur clôture. Ainsi, lorsqu'un test échouait il était possible, d'un simple clique sur TestLink, de générer automatiquement un ticket sur Mantis avec toutes les informations rajoutées précédemment. Les développeurs avaient donc toutes les données (url, service etc...) pour reproduire l'anomalie en locale et la corriger. \\
	
	TestLink permet de personnaliser la génération des rapports de tests après l'exécution d'un plan complet. De plus, il est possible de générer le rapport dans différents formats. Nous avons donc décidé que chaque livraison comporterait un rapport complet contenant toutes les informations au format PDF. Cependant celui-ci pouvant être très conséquent, nous avons choisi de l'accompagner d'un rapport léger généré au format excel ne contenant que le nom des tests et leur statut (succès, echec, bloqué) avec un code couleur permettant a PBI de rapidement repérer les tests en échec et leur nombre. Ces derniers ont déclaré être très satisfait des nouveaux rapports fournis, c'est pourquoi il a été décidé que l'exécution des tests fonctionnels sur la plateforme TestLink serrait obligatoire avant chaque livraison. En attendant l'attribution d'un serveur pour héberger le gestionnaire de test, celui-ci a été mis en place sur le serveur personnel d'un des architecte du projet afin de le rendre accessible à toute l'équipe. \\
	
	La réalisation de ces tests et leur exécution m'a permis de relever certains écarts entre le produit réalisé et les spécifications. Par exemple, le service LoanDetails ne fournissait pas, dans sa réponse, certains champs demandé par le client. C'est pourquoi, après avoir centralisé tous les écarts que j'avais relevé, j'ai organisé une réunion avec mon chef de projet ainsi que certains développeurs dans le but de déterminer lesquels pourraient être corriger et lesquels ne le pourraient pas, nécessitant de prendre contact avec le client afin de clarifier la situation. Au terme de cette réunion, j'ai pu créer une nouvelle version des spécifications afin de les mettre à jour puis j'ai fait part de ces modifications à PBI à travers des mails rédigés en anglais. Ensuite, la mise en place de ces outils a permi à l'équipe de pouvoir effectuer les livraisons dans de meilleurs conditions puisque les anomalies pouvaient être détectées avant d'attendre les retours du client. De plus, tous les tests étaient centralisés, organisés et pouvaient être exécutés en parallèle par différentes personnes ce qui permettait un gain de temps à la fois sur l'exécution des tests mais aussi sur leur gestion (archivage, formalisme, perte de documents etc...). \\
	
	Néanmoins, afin d'exécuter les tests, les développeurs utilsaient Postman pour construire et envoyer les requêtes. Si cet outil est extrèmement utile pour développer un nouveau service ou tester un nouveau endpoint lors du développement, il n'est pas conçu pour exécuter un grand nombre de requête les unes après les autres à des fins de tests. En effet, comme nous l'avons expliqué dans la partie \ref{axway} les requêtes doivent posséder un token d'authentification dans leur header pour espérer passer la gateway et atteindre la couche microservices. Or, pour cela, il est nécessaire d'envoyer, toujours avec Postman, une requête de génération de ce token à l'API gateway. Après cela, il faut se connecter sur l'interface web fourni par Axway, retrouver la requête ainsi que sa réponse et copier le token. Cependant, pour accéder à cette interface il faut d'abord être en mesure d'accéder à l'environnement testé, Homo3 ou RGB, qui, rappelons le, n'ont pas les mêmes instances de l'API Gateway et ne sont pas accessibles de l'extérieur. Pour cela, il faut se connecter en RDP sur TSE puis ensuite utiliser les bonnes adresses et identifiants pour accéder à l'interface désirée. De retour sur Postman, il faut coller le token dans le header de la requête que l'on souhaitais tester. Cette démarche est illustrée sur l'annexe \ref{a3} qui décrit la procédure d'authentification gérée par la gateway. Et il faut répéter cette démarche \textbf{à chaque fois que le token expire}, ce qui ne pose pas de problèmes lorsque l'on souhaite envoyer une requête pour tester son code mais devient rapidement très fastidieux lorsque l'on a des centaines de requêtes à envoyer pour exécuter tous les tests. 
	Un exemple classique serait de tester le service LoanDetails. Pour cela il faut générer un token pour un abonné. Ensuite, on appelle le service ClientList pour afficher les produits de l'abonné puis on regarde si celui-ci possède un produit de type \textit{loan} pour récupérer son id (crédit en français). Enfin, on appelle le service LoanDetails avec l'id du loan obtenu précédemment. Si le client ne possède pas de loan il faut recommencer le processus avec divers abonnés jusqu'à trouver ce que l'on cherche. Maintenant, certains services nécessitent d'en   appeler trois autres différents ce qui implique de faire de nombreuses combinaisons avant de tomber sur celle qui nous permet de réaliser le test. Il résulte de cela une perte considérable de temps qui aurrait pu être mis au profit du développement, de l'amélioration des points jugés sensibles ou encore de la correction des anomalies. En effet, il n'était pas rare que l'ensemble des tests puissent occuper une personne presque \textbf{une demi journée}, ce qui se révèle être énorme sur des sprints de deux ou trois semaines. Il fallait donc trouver le moyen de conserver la procédure de test et la génération de rapports tout en réduisant drastiquement le temps que cela demandait, c'est pourquoi nous avons décidé d'automatiser les tests fonctionnels.
	
	
	
\section{Gestion des doubles relations}
	\subsection{Définition du besoin}	
	
	La version 1.0 de l’application mobile est réservée à la clientèle \textit{private} (on parle aussi de personnes physiques PP) de NOBC. Cependant, Neuflize ayant la volonté d'étendre la portée de ses services numériques afin de toucher un public toujours plus large, elle a souhaité rendre l'application disponible pour les personnes morales (PM). Ainsi, elle souhaitait inclure très rapidement dans une version 1.1 la clientèle \textit{double relation}, c’est-à-dire la clientèle de type "entreprise" mais qui dispose en même temps dans ses accès des produits "particuliers"; typiquement un entrepreneur qui en plus de ses comptes "entreprises" peut aussi voir/faire de la transaction sur ses comptes "privés". Or, d'un point de vue légal, les entreprises n'ont pas les mêmes droits que les clients private c'est pourquoi des modifications ont dû être apportées aux API. Les méthodes agiles nous apportant une certaine flexibilité, nous avons organisé une réunion avec les métiers afin d'étudier le besoin en détail et déterminer quelles modifications seraient appropriées. \\
	
	A l'issue de cette réunion avec le client, nous avons émis différentes problématiques afin de déterminer les actions à mettre en place au niveau de la couche microservices. Sur le site internet, un abonné double relation dispose d’un bouton lui permettant de visualiser soit ses comptes "entreprises" soit ses comptes "privés". Néanmoins, il n’est pas prévu à ce jour de tel bouton permettant à l’abonné de passer de ses comptes "entreprises" à ses comptes "privés" et inversement sur la version 1.0 de l'application mobile. En effet, celle-ci est réservée à la clientèle "private". De ce fait les produits (comptes, cartes, crédits, titres, etc.) à présenter doivent être des produits "private". De plus, EFS s’est engagé à fournir toutes les informations dans ses API pour que le niveau API microservices puisse faire la distinction entre les comptes et produits "entreprises" et les comptes et produits "privés". Ainsi, le besoin au niveau de la couche API microservices consistait donc à mettre en place des filtres pour présenter/filtrer les données de/vers la couche API de PBI.
	
\subsection{Mise en œuvre}
	
	Avant d'aller plus loin, il est nécessaire de définir ce que sont les \textit{racines}. \hl{TODO : racines} \\
	
	La mise en place de la gestion des doubles relations m'a été confiée et je me suis donc chargé de construire la solution. J'ai d'abord présupposé que les actions menées par EFS avaient été réalisées. Afin de les définir de manière précise, j'ai organisé un point avec une personne métier. Cette dernière a donc pu me les expliciter et elles étaient les suivantes :
	\begin{itemize}
		\item Véhiculer l’information <type de racine> au niveau du back office EFS
		\item Enrichir l’API pour ajouter l’information <type de racine>
		\item Modifier l’API pour remonter les abonnés dont le profil est "particulier" ou "particulier bourse" ainsi que les abonnés dont le profil est "entreprise" ou "entreprise bourse" et qui ont des racines typées "privé" \\
	\end{itemize}
	
	De plus, une table de décision a été mise en place côté EFS (en SQL) dans le but de décider si un type de racine est de type entreprise ou de type particulier afin de le remonter ou non depuis EFS et de préciser l’ordre de présentation (colonne "poids") à exploiter le cas échéant. Cette table était de la forme suivante (les termes métiers importent peu pour la compréhension): \\
	
\begin{table}[h!]
	\center
	\begin{tabular}{| c | c | c |}
     \hline
     Type de racine & Entreprise/Particulier & Poids \\ \hline
     PARF & Particulier & 1\\ \hline
     PARH & Particulier & 2\\ \hline
     STE & Entreprise & 10\\ \hline
     ... & ... & ...\\
     \hline
	\end{tabular}
	\caption{Table de décision pour les doubles relations}
	\label{tableDecisionRacine}
\end{table}
	
	Les règles, issues des spécifications, à respecter concernant l'identification du type des racines étaient les suivantes :
	\begin{itemize}
		\item Si une racine est liée à plusieurs produits dont des comptes :
			\begin{itemize}
				\item Son type entreprise/particulier est porté par la nature du compte
				\item Tous les autres produits (crédit, titre, carte bleue, assurance vie) rattachés à cette racine auront la même nature que le compte
			\end{itemize}
		\item Si une racine est liée à plusieurs produits sans comptes :
			\begin{itemize}
				\item L’API côté EFS devra transmettre l’origine : P (particulier), E (entreprise)
				\item Cette API est cependant susceptible de transmettre <blanc> : alors ce sera le type de la racine qui déterminera si l’ensemble de ses produits est de type particulier ou entreprise
			\end{itemize}
		\item Le cas où une racine est liée à un compte entreprise et un compte particulier n’est pas considéré et est traité comme une anomalie à corriger \\
	\end{itemize}
	
	En outre, un cahier des charges m'a été fourni dans lequel il était stipulé toutes les modifications à effectuer concernant l'appel des services EFS. En effet, comme nous l'avons déjà expliqué, nos services consomment ceux d'EFS afin de la faire de la composition. 
	
	Ainsi, il était par exemple expliqué que le service EFS permettant d'obtenir les assurances vie liées à une racine ne devrait être appelé uniquement pour les clients ne possédant que des produits privés (pas entreprises). 
	
	Pour un autre exemple, EFS expose un service permettant de récupérer les informations de toutes les cartes bleues liées à une racines. Aucun changement n'était demandé concernant l'appel de ce service mais il était nécessaire de filtrer les cartes obtenues afin de ne garder que celles liées à un profil privé et non entreprise. 
	
	Mon objectif était de déterminer quels seraient les microservices impactés et dans quelle mesure par la gestion des doubles relations. Après analyse du cahier fourni, j'ai synthétisé l'ensemble des informations dans le diagramme présent sur la figure \ref{doubleRelation}. \\
	
\begin{figure}[h!]
	\includegraphics[scale=0.45]{images/travailNeuflizeOBC/doubleRelation/doubleRelation.png}
	\centering
	\caption{Objets impactés par les doubles relations}
	\label{doubleRelation}
\end{figure}

	Il était maintenant possible de déterminer les services qui aurraient besoin de développement supplémentaire. D'après ce schéma, les services concernés étaient ceux présent dans le tableau figure \ref{servicesDoubleRelation}. Une fois cela établi il ne restait plus qu'à procéder au développement et à effectuer les modifications nécessaires afin d'assurer la gestion des doubles relations dans la couche API microservices. \\ 
	
\begin{table}[h!]
	\center
	\begin{tabular}{| c | c | c |}
     \hline
     Service & Raison \\ \hline
     accountToDebit & Créer un filtre sur les comptes émetteurs \\ \hline
     addressBook & Créer un filtre sur les comptes bénéficiaires internes \\ \hline
     transactionOverview & Créer un filtre sur les ordres de transaction\\ \hline
     clientList & Créer un filtre sur les racines récupérées \\ & et les trier selon leur poids\\ \hline
     accountOverview & Créer un filtre sur les cartes bleues \\
     Tous & Rajouter le paramètre "codeProfil=PP" dans les appels aux services EFS \\ 
     \hline
	\end{tabular}
	\caption{Modifications par services pour la gestion des doubles relations}
	\label{servicesDoubleRelation}
\end{table}

\subsection{Résultats}

	Comme nous l'avons vu, les modifications à apporter sont mineures et les nouveaux développements sont loin d'être conséquents, mais ces derniers impactent de très nombreux endroits du code et nécessite d'altérer plusieurs services ce qui aurraient pu gêner les autres développeurs dans leur travail. De plus, EFS n'étant pas prêt de son côté, j'ai travaillé en utilisant \textit{WireMock}, un outil permettant de créer des \textit{mocks} qui sont des objets simulés reproduisant le comportement d'objets réels. Ainsi, cela m'a conforté dans l'idée de créer une nouvelle branche sur le github du projet sur laquelle j'ai poussé le code permettant la gestion des doubles relations. Une fois EFS prêt, il sera alors facile de simplement réaliser un merge de cette branche avec la branche master. \\
	
	Bien que la réalisation de cette tâche ne présentait pas de grandes difficultés d'un point de vue de technique, celle-ci m'a permis de me familiariser avec l'aspect relationnel du projet. En effet, nous travaillons en employant des méthodes agiles ce qui m'a conduit a rencontrer le client c'est-à-dire les métiers travaillant dans le même openspace que nous. Ceci m'a permis d'avoir une première expérience complète, bien qu'à petite échelle, allant de la définition du besoin avec le client jusqu'à la réalisation de la fonctionnalité en accompagnant celui-ci tout au long du processus via diverses réunions afin de fixer les points cruciaux. Ensuite, j'ai pu réaliser cette tâche de manière autonome et donc eu la chance de pouvoir mettre en place mes propres solutions tout comme pour la partie concernant les tests fonctionnels, à ceci prêt que dans ce cas il ne s'agissait pas d'un travail indépendant mais en relation direct avec le code source et une livraison, impliquant une certaine confiance.
	En outre, j'ai été contraint d'apporter des modifications à travers toute l'API ce qui m'a permis de pouvoir mieux m'approprier le code et de monter ainsi en compétence sur la stack utilisée et surtout sur le développement de "microservices".
	
\section{Dashboard}
	\subsection{Cachier des charges}
		La mise en production de l'application mobile approchant, le client a émis le souhait de pouvoir collecter des données sur la manière dont les utilisateurs l'utiliseraient. Il souhaitait avoir à sa disposition un outil lui permettant de monitorer l'application afin de pouvoir évaluer la satisfaction des utilisateurs et de relever les axes d'amélioration. 
	
	Par exemple, pour réaliser une transaction bancaire via l'application mobile, l'utilisateur doit se connecter puis initier la transaction en choisissant des comptes émetteurs/receveurs ainsi qu'un montant. Après cela, un clavier virtuel apparait sur lequel il doit rentrer son code secret, ce qui aura pour effet de \textit{signer} la transaction. Une des requêtes de Neuflize était de pouvoir avoir un ratio entre le nombre de transactions initiées et le nombre de transactions signées afin de déterminer les nombres de transaction abandonnée en cours de création et donc si ce système de signature ne rebutait pas les utilisateurs dans leur démarche. En effet, un grand nombre de transaction initiée mais non signée pourrait indiquer que cette procédure ne plait pas aux clients de la banque parce qu'elle serait trop longue ou peu pratique. De plus, cela permettrait aussi de suivre les potentielles erreurs qui apparaitraient entre l'initiation et la signature afin de prendre des mesures adéquates. \\
	
	Afin de suivre ces informations, Neuflize souhaitait pouvoir disposer d'un \textit{dashboard} (tableau de bord), qui centraliserait l'ensemble des données collectées sous la forme de graphiques. Le dashboard contiendrait alors tous les graphiques et permettrait de naviguer facilement afin de retrouver les informations recherchées, et ce en fonction du temps. En effet, le critère de temps est indispensable dans l'optique de suivre l'évolution de l'application et pour situer les événements.
	
	Ainsi, avant que la réalisation d'un tel dashboard me soit confiée, les outils qui seraient utilisés avaient été décidé suite à une réunion entre l'un des architectes du projet et les métiers. Il m'a donc été demandé de répondre à ce besoin en mettant en place, dans un premier temps, la stack ELK (Elasticsearch, Logstash et Kibana) dont nous parlerons dans la partie suivante puis en créant une première ébauche des graphiques en local à l'aide de données factices. En outre, une deadline avait été fixée puisque le dashboard, ou du moins la collecte des données, devait être prêt pour la mise en production de l'application. En effet, les premières informations sont essentielles pour déterminer les évolution à apporter à l'application et permettent d'avoir une idée de la satisfaction des utilisateurs. \\
	
	Il m'a été remis un document réalisé par Neuflize, dont un extrait est disponible figure \ref{elkBesoin}, dans lequel il était décrit l'ensemble des besoins c'est-à-dire toutes les informations qui étaient souhaitées, comment elles devaient être traitées et la manière dont elles devaient être gérées. Le premier travail à effectuer a été de déterminer quelles informations pouvaient être remontées par l'API microservices et lesquelles ne le pouvaient pas. Par exemple, nous pouvons voir qu'il était demandé le nombre de transaction initiée sur une période donnée, chiffre qu'il est possible pour nous de fournir. Cependant, il est aussi demandé le temps passé sur chaque écran de l'application par l'utilisateur ou encore le nombre de page que celui-ci a visité, ce qui est impossible à fournir de notre côté. En effet, la couche API microservices est bien évidemment située dans la partie backend du projet et est une API REST exposant des services. Ainsi, il nous est totalement impossible d'obtenir des informations si nos services ne sont pas appelés. Le temps passé sur chaque écran n'est pas calculable puisque certains écrans peuvent ne pas appeler de services, certains en appelleront plusieurs etc... Il a donc fallu analyser l'ensemble des besoins dans le but de déterminer ce qui était réalisable de notre côté, quant au reste, il était possible de l'obtenir du côté frontend via d'autres outils qui avaient été mis en place par l'équipe en charge de cette partie tel qu'\textit{Omniture}.
	
\begin{figure}[h!]
	\includegraphics[scale=0.6]{images/travailNeuflizeOBC/dashboard/elkBesoin.png}
	\centering
	\caption{Extrait du besoin client pour le dashboard}
	\label{elkBesoin}
\end{figure}
	
	Une fois cette étude terminée j'ai effectué un rapport auprès d'un architecte du projet qui a validé mon analyse puis j'ai organisé une réunion avec les métiers afin de clarifier la situation et leur faire savoir tout ce qui était faisable par l'API microservices. Au terme de cette réunion et après discussion sur la manière de remplacer certaines informations inaccessibles par d'autres s'en rapprochant ou permettant de les retrouver, j'ai pu me lancer dans la mise en place de la stack ELK \cite{bib_elk} et la réalisation du premier jet du dashboard.

\subsection{La stack ELK : ElasticSearch, Logstash et Kibana}
		Comme nous l'avons dit précédemment, pour réaliser le dashboard il a été décidé que la stack ELK serait utilisée. J'ai donc, dans un premier temps, du me former à l'utilisation de cette dernière qui est constituée de trois produits de la société Elastic que sont Elasticsearch, Logstash et Kibana \cite{bib_elk2}. Son objectif premier est de stocker, d'analyse et de lire les logs générer par un projet tel qu'une API REST dans l'optique de faire du monitoring, de suivre l'activité d'un service ou de prévenir des dysfonctionnements. Nous allons maintenant décrire en quoi chacun des outils est intervenu dans le processus de réalisation du dashboard. Pour cela, il est possible de se référer à la figure \ref{elk} afin de pouvoir situer chaque composant.

	\subsubsection{Elasticsearch}
	La quantité de logs générés par une application de cette ampleur peut rapidement devenir très conséquente et de nombreuses problématiques sont alors soulevées concernant le stockages des messages importants, leur recherche ou leur analyse. Les bases données relationnelles classiques ont montré leurs limites concernant le travail avec de tels volumes de données sur le long terme que ce soit pour le stockage ou les temps de réponse lors d'une recherche, et ce malgré des requêtes bien construites avec des jointures judicieuses entre les tables. Elasticsearch a pour objectif de répondre à cette problématique. \\
	
	Il s'agit d'un moteur d'indexation, de stockage et de recherche de données développé en Java basé sur \textit{Lucene}, une bibliothèque d'indexation et de recherche de texte créée par la fondation Apache. Son principal atout est qu'il permet de stocker de large volume de données tout en permettant aux utilisateurs d'effectuer des recherches en temps réel. Les données sont stockées sous la forme de \textit{document}. Il s'agit d'une unité basique d'information qui peut être indexée. Les index sont des collections de documents possédant des caractéristiques similaires. Les index peuvent aussi posséder un type afin de les diviser en plusieurs parties "logiques". Il est possible, par exemple, de créer un index \textit{api-microservices-2017.07} avec un type \textit{log}. 
	
	Nous pouvons alors récupérer le document d'id 1 avec la requête suivante :	

\begin{lstlisting}[language=json]
 GET api-microservices-2017.07/log/1?pretty
 \end{lstlisting}
	
	La réponse contiendrait ledit document avec les champs définis par nos soins (ici timestamp, logLevel, service et message) :
	
\begin{lstlisting}[language=json]
{
 "_index": "api-microservices-2017.07",
 "_type": "log",
 "_id": "1",
 "_version": 1,
 "found": true,
 "_source" : {
  "timestamp": "2017-07-07T14:15:45",
  "logLevel": "INFO",
  "service": "account-service"
  "message": "elasticsearch example"
 }
}
	\end{lstlisting}
	 
	 Dans notre cas, Elasticsearch \cite{bib_elastic} est utilisé comme support de stockage qui sera par la suite interrogé par Kibana. Cependant, il peut tout à fait être interroger depuis une API, c'est pourquoi il ne dispose pas de client dédié.
	
	\subsubsection{Logstash}
	Logstash \cite{bib_logstash} est un \textit{ETL} ou \textit{Extract-transform-load}. Cela désigne les outils capable de synchroniser des volumes de données massif depuis une source précise vers une autre. Ainsi, Logstash agit un peu à la manière d'un pipeline capable de prendre un grand nombre de messages en entrée depuis plusieurs sources pour les traiter, effectuer des transformations ou encore les filtrer avant de les renvoyer vers un support de stockage. Le schéma figure \ref{logstash} explicite son principe de fonctionnement. La liste des technologies mentionnées dans celui-ci n'est pas exhaustive, il s'agit purement d'exemples de ce qu'il est possible de faire à titre démonstratif. \\
	
	\begin{figure}[h!]
		\includegraphics[scale=0.4]{images/travailNeuflizeOBC/dashboard/logstash.png}
		\centering
		\caption{Logstash}
		\label{logstash}
	\end{figure}
	
	Logstash accepte en entrée un nombre impressionnant de formats différents comme tout ce qui est représenté sous forme de chaîne de caractères mais aussi les événements issus de sources différentes et ce de manière simultanée. Dans notre cas, logstash prendra en entrée des fichiers logs fournit par Filebeat dont nous parlerons plus en détails peu après. Ces logs auront été générés au préalable par l'API microservices à l'aide de \textit{Logback}. \\
	
	Une fois les données acquises, celles-ci sont analysées puis transformer par les filtres de Logstash. Il est ainsi possible d'ajouter, modifier ou supprimer de l'information, extraire des valeurs depuis des messages, analyser des événement et bien d'autres. Le filtre le plus puissant mis a disposition par Logstash est \textit{Grok} que nous utiliserons ici et qui permet de transformer dynamiquement des données non structurées en données structurées en découpant une ligne de log en un ensemble de champs prédéfinis. Nous pourrons donc utiliser ce filtre afin de structurer les informations de telle manière qu'elles soient facilement exploitable par Kibana. Par exemple, la ligne de log suivante :
	
\begin{lstlisting}[language=json]
 [2017-07-07T14:15:45] [INFO] [account-service] - elasticsearch example
\end{lstlisting}
	
	peut être transformer sous la forme du document structuré montré en exemple plus haut à l'aide d'un filtre adéquat définissant les champs (comme timestamp ou loglevel). \\
	
	Enfin, une fois les données formatées, il est possible de les envoyer vers la destinaiton de notre choix, et là encore Logstash offre la possibilité de transmettre les données vers un grand nombre de sources différentes. Il sera bien évidemment ici connecté à Elasticsearch et permettra donc de stocker les logs de manière structurée et indexée au sein de la base de données. En outre, il offre une grande flexibilité puisque chacune des entrées/sorties ou mêmes les filtres sont implémentés sous la forme de plugin. Par ailleurs, la société Elastic a mis à disposition une API permettant de faciliter le développement de ces plugins, ce qui permet d'ajouter de nombreuses nouvelles possibilités à Logstash.
	
	\subsubsection{Kibana}
	Avec les outils évoqués précédemment, nous avons de quoi stocker nos logs, les analyser, les transformer pour structurer les informations qu'ils contiennent et rechercher ce que nous souhaitons parmi eux à l'aide de requêtes Elasticsearch. Cependant, cela n'est absolument pas "user friendly" et nécessite des connaissances techniques ce qui ne suffit pas à satisfaire le besoin client. Kibana \cite{bib_kibana} permet palier cette faiblesse en nous fournissant une interface nous permettant de choisir la mise en forme de nos données. Celui-ci permet de construire des graphiques, nommés \textit{visualisations}, tels que des histogrammes, des camemberts ou encore des tableaux. Ensuite, Kibana permet de choisir une plage de date pour ne traiter que les données situées dans cette dernière. Une fois le dashboard et les visualisations créés, il suffit de changer le temps et tous les graphiques s'actualiseront automatiquement ce qui permet de facilement retrouver des informations. Par exemple, il suffit d'un simple clic pour connaitre le nombre de connexion qu'il y a eu hier, le mois dernier ou entre le 23 juillet et le 15 août. De plus, Kibana vient avec une nouvelle composante nommée \textit{Timelion} permettant d'effectuer des analyses chronologiques approfondies. \\
	
	Ici, Kibana alimentera les visualisations avec les données stockées dans la base d'Elasticsearch et sera l'unique interface homme machine utilisée. En effet, il offre une console développeur permettant de faciliter la construction des requêtes Elasticsearch et permet de visualiser les données stockées en base à l'aide une option \textit{discovery} affichant pour chaque ligne de log brute les champs structurés définis dans logstash.
	
	\subsubsection{Filebeat}	
	Elastic a mis à notre disposition, en plus de la stack ELK, des agents de transfert nommés les \textit{Beats} \cite{bib_filebeat}. Il en existe pour tout type de données mais dans notre cas nous utiliserons Filebeat conçu spécialement pour le transfert des logs et fichiers. Celui-ci permet de collecter l'ensemble des logs, qu'importe où ils se trouvent, pour les envoyer ensuite à Logstash. Il est possible de le configurer pour qu'il ne transfert pas les fichiers de logs entiers mais seulement les lignes qui nous intéressent. Celui-ci ne nécessite aucune intervention humaine une fois qu'il est mis en place, à chaque ligne de log générée il en est informé et effectue le transfert automatiquement. En outre, si Logstash est occupé et possède déjà un gros volume de données à traiter, Filebeat en sera également informé et pourra ralentir son flux d'envoi afin d'éviter tout problème de congestion.

	\subsubsection{Les interactions en résumé}
	Toutes les interactions entre les différents composants sont résumées sur le schéma figure \ref{elk}. Pour commencer, les logs sont générés par l'API microservices grâce à Logback qui permet de configurer quelle ligne ira dans quel fichier et où ces derniers seront créés. Dès que des lignes de logs sont générées, Filebeat est averti et va donc lire le fichier de logs concerné, détecter les nouvelles lignes puis les envoie à Logstash si elles remplissent les critères. Ce dernier récupère ces lignes non structurées en entrée et agit comme un pipeline qui produira en sortie des données structurés à l'aide de champs prédéfinis. Ensuite, Elasticsearch reçoit ces données de la part de Logstash, les indexe puis les stocke en base de données. Enfin, Kibana interroge Elasticsearch à l'aide de requêtes en fonction du temps sélectionné pour alimenter les visualisations créées au sein du dashboard. 
	
\begin{figure}[h!]
	\includegraphics[scale=0.5]{images/travailNeuflizeOBC/dashboard/elk.png}
	\centering
	\caption{Interactions au sein de la stack ELK}
	\label{elk}
\end{figure}
	
\subsection{Réalisation des graphes}
		Comme nous l'avons vu dans la partie précedente, les données qiu seront exploitées sont des logs. Ainsi, après avoir installé et configuré les différents éléments de la stack ELK, mon objectif était de générer des logs afin que puisse tester chacune des composantes entrant en jeu. Cependant, avant d'aller plus loin, il faut savoir que l'API microservices génère déjà de nombreux logs informatif permettant d'analyser les actions effectuées, l'état des services ou encore les erreurs. De plus, il existe d'autres outils comme RabbitMQ, un message broker, qui génère une grande quantité de logs. Les lignes qui nous intéressent pour construire les visualisations sont donc noyées dans un flot d'information qui ne cesse de croître à mesure que le temps s'écoule. La première étape a donc été de définir un modèle permettant de différencier les lignes de logs qui seraient destinées au dashboard de celles qui ne le sont pas. \\

\begin{itemize}
	\item definission du modele analytics
	\item création de données bidons pour les logs
	\item conf logstash filtre grok
	\item requête elasticsearch
	\item creation des visualisation sur Kibana
	\item limitations
\end{itemize}
	
\subsection{Recette et déploiement}
		Le dashboard étant maintenant terminé, une phase de recette a été organisée avec deux métiers ainsi que mon chef de projet afin que je puisse présenter le travail que j'avais effectué. Des extraits sont disponibles en annexe \ref{c2}. Tous les graphiques ont été passés en revue et les métiers ont pu vérifier que chacun des besoins que je m'étais engagé à satisfaire était effectivement présent sur ledit dashboard. Au terme de cette réunion, le dashboard a été approuvé malgré les quelques spécifications qui n'avaient pas pu être satisfaites et nous avons donc décidé de le mettre en place sur tous les environnements à savoir Homo3, RGB et la production. Nous avions déjà schématisé ces environnements figure \ref{environnement} mais ces derniers étaient incomplets. En effet, en réalité, chacun d'entre-eux possède différentes VM dont une dite de \textit{supervision} contenant les outils de monitoring et de supervision et une appelée \textit{microservice} sur laquelle se trouve le code source de l'API microservices. Les environnements RGB et de prod possèdent même deux VM microservices pour le load-balancing. La stack ELK a été installé sur supervision alors que FileBeat a été installé sur microservices (a et b) puisqu'il devait récupérer les logs générés par l'API. J'ai eu la chance de pouvoir procéder moi-même à l'installation de tous les éléments sur RGB. Après cela, je me suis occupé de la maintenance de ces outils afin de corriger les éventuels problèmes qui apparaissaient.\\
	
\begin{figure}[h!]
	\includegraphics[scale=0.45]{images/travailNeuflizeOBC/dashboard/elkDeploiement.png}
	\centering
	\caption{Architecture de la stack ELK sur tous les environnements}
	\label{elkDeploiement}
\end{figure}

	J'ai décrit jusqu'ici les principaux travaux que j'ai mené chez Neuflize OBC. Cependant, il reste de nombreuses tâches que j'ai réalisé comme la rédaction d'un guide complet concernant le projet destiné à l'équipe de maintenance qui s'occupera prochainement de nos API puisque celles-ci sont passées en production. J'ai aussi procédé à la réalisation de certaines tâches dites "architectes" via la mise en place de cache grâce à \textit{Redis} ou encore la configuration de RabbitMQ, le message broker, sur les différents environnements. Il s'agissait de petites tâches que j'ai pu mener conjointement avec l'un des architecte du projet, ce qui m'a permis de découvrir une facette de plus du projet (adminsitration via les consoles, TSE etc...) Nous allons maintenant nous intéresser plus en détails au travail que j'ai réalisé chez la Banque Privée 1818.

\section{Mise en production}
	\begin{itemize}
	\item Réalisation du guide ultime \# doc
	\item Maintenance des dashboards
	\item Tâches archi :
		\begin{itemize}
			\item Paramétrage de RabbitMQ sur tous les environnements
			\item Requêtes séquentielles sur zipkin > paralléliser
			\item Plusieurs fois la même requête EFS sur Zipkin > mise en cache avec Redis
		\end{itemize}
\end{itemize}
%-----------------------------------------------------------------------------------------------------------------
%-----------------------------------------------------------------------------------------------------------------
	
%-----------------------------------------------------------------------------------------------------------------
%-----------------------------------------------------------------------------------------------------------------

\chapter{Banque Privée 1818}

	\section{Environnement de développement}
	\input{src/travailBP1818/environnement}
	
\section{Sprint 5 : MIF : "Objectifs financiers"}
	\input{src/travailBP1818/objectifsFinanciers}
	
\section{Sprint 6 : Restitution du scoring}
	\input{src/travailBP1818/scoring}
	
%-----------------------------------------------------------------------------------------------------------------
%-----------------------------------------------------------------------------------------------------------------
	
%-----------------------------------------------------------------------------------------------------------------
%-----------------------------------------------------------------------------------------------------------------
	
\chapter*{Conclusion et perspectives} %même mise en page chapitre mais numérote pas
	\addcontentsline{toc}{chapter}{Conclusion et perspectives} %ajout au sommaire
	
	\newenvironment{changemargin}[2]{%
\begin{list}{}{%
\setlength{\topsep}{0pt}%
\setlength{\leftmargin}{#1}%
\setlength{\rightmargin}{#2}%
\setlength{\listparindent}{\parindent}%
\setlength{\itemindent}{\parindent}%
\setlength{\parsep}{\parskip}%
}%
\item[]}{\end{list}}

\begin{changemargin}{-1cm}{-1cm}

	Ce stage a été, pour moi, extrêmement enrichissant de par la richesse des tâches que j'ai pu effectuer. En effet, j'ai eu la chance de pouvoir travailler à mi-temps sur deux projets importants de l'agence 512, plus grosse agence sur secteur bancaire de Sopra Steria. Ces projets digitaux sont actuellement des modèles ainsi que des portes d'entrées vers de nouveaux projets et clients voulant franchir le pas de la transformation numérique. Sur ces projets j'ai pu à chaque fois travailler au sein d'équipes de développement dynamiques intégrées dans leur environnement et centrée autour de la collaboration pour la réussite d'un projet digital dans son ensemble et non pas pour la simple réalisation d'un produit à vendre. Aussi, de par l'utilisation des méthodes agiles et des relations ainsi créées j'ai pu sortir du cadre développeur classique afin de m'entretenir directement avec le client, que ce soit pour revoir des spécifications et le conseiller comme sur le sujet des dashboards, pour des recettes ou encore pour que celui-ci m'éclaircisse son point de vue et son besoin. Cela permet d'avoir une idée nettement plus claire de ses attentes mais aussi des contraintes auxquelles il doit faire face et donc d'être force de proposition dans le but de lui venir en aide. Il s'agit là d'une véritable plus-value pour mon insertion professionnelle puisque j'ai eu la chance de travailler et d'interagir non seulement avec des développeurs mais aussi avec des architectes, des équipes fonctionnelles, de pilotage, de conduite du changement, des commerciaux et bien sûr des équipes métiers. \\
	
	Durant mon stage j'ai eu la chance d'être intégré à la "communauté architecture" de la division banque et finance de Sopra Steria dans laquelle une veille technologique est assurée et des brown bag lunch (BBL) sont organisés. Il s'agit de présentations sur des sujets technologiques variés effectuées sur l'heure du repas par des développeurs ou des architectes membres de la communauté. Cela apporte une fois encore un côté humain favorisant les relations tout en permettant de rester centré autour des métiers de l'information. Les architectes essaient de présenter les dernières technologies et n'hésitent pas à les employer dans leurs projets actuels. Par exemple, chez Neuflize OBC, une architecture microservices a été mise en place, une première dans la division bancaire Sopra Steria. Le succès de cette dernière a pu être transmis par le biais de la communauté et une étude est menée en ce moment par des développeurs de chez BForBank pour l'utiliser sur l'un de leur projet. De plus, les outils utilisés sont ceux de la stack Netflix, récents et open-source. Du côté de BP1818, nous avons recours à Angular2 pour construire notre frontend et nous avons même migré sur Angular4 sorti en version stable au mois de mars 2017, puis Angular5 en juillet. \\
	
	Un autre point fort que j'ai pu relever durant ce stage concerne la richesse des travaux que j'ai pu réaliser. Du côté de BP1818 j'ai pu effectuer du développement aussi bien sur la partir backend que sur la partir frontend. Une certaine liberté m'a été laissé puisque j'ai souvent pu réaliser mes tâches de manière autonome tant au niveau conception que développement bien qu'étant sous la supervision de mon tuteur en cas de nécessité. Chez Neuflize OBC, je n'ai effectué que peu de développement mais j'ai pu m'intéresser à la partie tests fonctionnels de par leur réalisation et la mise en place d'outils pour les automatiser et faire gagner du temps à l'équipe. J'ai aussi pu mettre en place du monitoring applicatif, sujet durant lequel je pouvais moi-même organiser des points avec les métiers. De plus, une fois mes sujets terminés et avant mon passage à temps plein chez BP1818 l'un des architectes m'a permis d'effectuer des tâches sur les environnements de pré-production et de production comme l'installation de la stack ELK, la configuration RabbitMQ ou encore la résolution de certains problèmes d'infrastructure. Cela m'a apporté de nombreuses connaissances, une importante stack et m'a encore une fois permis de sortir de la seule vision "développeur". \\
	
	Fort de cette expérience, j'ai accepté l'offre d'emploi que Sopra Steria m'a proposé, les deux projets ayant comblés mes attentes et m'ayant permis de grandement m'épanouir. Le projet chez Neuflize OBC étant en phase de maintenance et l'application étant en production, je suis maintenant passé à temps plein chez BP1818. Comme nous l'avons déjà dit au début de ce rapport, cette banque est une filiale de Natixis Bank qui, étant satisfait du travail effectué, souhaite voir de nouveaux projets fleurir. Ainsi, Natixis Bank Luxembourg souhaite elle aussi mettre en place une application web très semblable au fronting digital de BP1818, projet qui pourrait être réalisé avec l'équipe actuelle qui connait bien le produit.
	
\end{changemargin}
%-----------------------------------------------------------------------------------------------------------------
%-----------------------------------------------------------------------------------------------------------------

%-----------------------------------------------------------------------------------------------------------------
%-----------------------------------------------------------------------------------------------------------------
% BIBLIOGRAPHIE
\begin{thebibliography}{9}
	\addcontentsline{toc}{chapter}{Bibliographie}	% permet de l'avoir dans le sommaire

%	\bibitem{Livre 01}
%		\textsc{Nom}, Prénom
%		\textit{Titre},
%		edition, date.
%
% USE : \cite{Livre 01}

\bibitem{bib_microservices}
		Article de \textsc{Martin Fowler - }\textsf{Explications sur l'architecture microservices} \\
		\url{https://martinfowler.com/articles/microservices.html}
		
\bibitem{bib_netflix}
		\textsc{Netflix OSS - }\textsf{Explications sur la stack Netflix OSS} \\
		\url{https://netflix.github.io/}

\bibitem{bib_zuul}
		Github de \textsc{Zuul - }\textsf{Documentation sur Zuul de la stack Netflix OSS} \\
		\url{https://github.com/Netflix/zuul}
		
\bibitem{bib_eureka}
		Github de \textsc{Eureka - }\textsf{Documentation sur Eureka de la stack Netflix OSS} \\
		\url{https://github.com/Netflix/eureka}

\bibitem{bib_hystrix}
		Github de \textsc{Hystrix - }\textsf{Documentation sur Hystrix de la stack Netflix OSS} \\
		\url{https://github.com/Netflix/hystrix}

\end{thebibliography}
%-----------------------------------------------------------------------------------------------------------------
%-----------------------------------------------------------------------------------------------------------------

%-----------------------------------------------------------------------------------------------------------------
%-----------------------------------------------------------------------------------------------------------------
% ANNEXES
\begin{appendices}

	\chapter{Architecture du projet - Neuflize OBC}
	
		\section{Services}
		\label{a1}
		\fancyhead[LE]{
	\begin{picture}(0,0) 
	\put(-30,-8){\includegraphics[width=49mm]{images/neuflizeOBCLogo.png}}
	\end{picture}
}
\fancyhead[LO]{
	\begin{picture}(0,0) 
	\put(-8,-8){\includegraphics[width=49mm]{images/neuflizeOBCLogo.png}}
	\end{picture}
}

\begin{table}[h!]
	\center
	\begin{tabular}{| c | c | c |}
     \hline
     Service & Fonction & Description \\ \hline
     Account-service & AccountOperations & Liste les opérations effectuées \\ & & sur un compte bancaire \\ \hline
     Account-service & AccountOverview & Liste les informations d'un compte \\ \hline
     Account-service & LoanDetails & Liste les informations d'un crédit \\ \hline
     Account-service & PositionDetails & Liste les informations d'une position \\ \hline
     Account-service & Rib & Liste les informations contenu dans un rib \\ \hline
     Profile-service & ClientList & Liste les informations relative au client et à ces racines \\ \hline
     Profile-service & TokenInfos & Liste les informations nécessaires à la génération d'un token \\ \hline
     Account-service & PortfolioAndLI & Liste les portefeuilles et assurances vie \\ \hline
     Account-service & RetrieveBalance & Liste les soldes d'un ou plusieurs comptes d'un client \\ \hline
     Transaction-service & AddressBook & Permet de lister les comptes créditeurs d'un client \\ \hline
     Transaction-service & NewTransfer & Permet d'initier une transaction \\ \hline
     Transaction-service & AccountToDebit & Permet de lister les comptes débiteurs d'un client \\ \hline
     Transaction-service & VKB & Permet de générer un clavier numérique virtuel \\ \hline
     Transaction-service & SignTransaction & Permet de signer numériquement une transaction \\ \hline
     Transaction-service & TransactionOverview & Liste les transactions effectuées par le client\\
     \hline
	\end{tabular}
	\caption{Fonctionnalités de l'API microservices}
	\label{fonctionnalites}
\end{table}

	
		\section{Architecture logicielle}
		\label{a2}
		\fancyhead[LE]{
	\begin{picture}(0,0) 
	\put(-30,-8){\includegraphics[width=49mm]{images/neuflizeOBCLogo.png}}
	\end{picture}
}
\fancyhead[LO]{
	\begin{picture}(0,0) 
	\put(-8,-8){\includegraphics[width=49mm]{images/neuflizeOBCLogo.png}}
	\end{picture}
}

\begin{table}[h!]
	\center
	\begin{tabular}{| c | c | c |}
     \hline
     Service & Fonction & Description \\ \hline
     Account-service & AccountOperations & Liste les opérations effectuées \\ & & sur un compte bancaire \\ \hline
     Account-service & AccountOverview & Liste les informations d'un compte \\ \hline
     Account-service & LoanDetails & Liste les informations d'un crédit \\ \hline
     Account-service & PositionDetails & Liste les informations d'une position \\ \hline
     Account-service & Rib & Liste les informations contenu dans un rib \\ \hline
     Profile-service & ClientList & Liste toutes les racines et comptes du client \\ \hline
     Account-service & PortfolioAndLI & Liste les portefeuilles et assurances vie \\ \hline
     Account-service & RetrieveBalance & \hl{TODO} \\ \hline
     Transaction-service & AddressBook & Permet de lister les comptes créditeurs d'un client \\ \hline
     Transaction-service & NewTransfer & Permet d'initier une transaction \\ \hline
     Transaction-service & AccountToDebit & Permet de lister les comptes débiteurs d'un client \\ \hline
     Transaction-service & VKB & Permet de générer un clavier numérique virtuel \\ \hline
     Transaction-service & SignTransaction & Permet de signer numériquement une transaction \\ \hline
     Transaction-service & TransactionOverview & Liste les transactions effectuées par le client\\
     \hline
	\end{tabular}
	\caption{Fonctionnalités de l'API microservices}
	\label{fonctionnalites}
\end{table}

	
	\chapter{Tests - Neuflize OBC}
		\section{Exemple de scénario de test}
		\label{b1}
		\fancyhead[LE]{
	\begin{picture}(0,0) 
	\put(-30,-8){\includegraphics[width=49mm]{images/neuflizeOBCLogo.png}}
	\end{picture}
}
\fancyhead[LO]{
	\begin{picture}(0,0) 
	\put(-8,-8){\includegraphics[width=49mm]{images/neuflizeOBCLogo.png}}
	\end{picture}
}

\begin{figure}[h!]
	\includegraphics[scale=0.6]{images/travailNeuflizeOBC/testsFonc/scenarioTest.png}
	\centering
	\caption{Scénario : Transaction bancaire}
	\label{scenarioTest}
\end{figure}

	\chapter{Monitoring applicatif - Neuflize OBC}
		\section{Visualisations des transactions}
		\label{c1}
		\begin{figure}[h!]
	\includegraphics[scale=0.5]{images/travailNeuflizeOBC/dashboard/kibanaTransaction_01.png}
	\centering
	\caption{Visualisations des transactions}
\end{figure}

\begin{figure}[h!]
	\includegraphics[scale=0.5]{images/travailNeuflizeOBC/dashboard/kibanaTransaction_02.png}
	\centering
	\caption{Visualisations des transactions}
\end{figure}
		\section{Extrait du dashboard final}
		\label{c2}
		\begin{sidewaysfigure}[ht]
	\centering
    \includegraphics[scale=0.5]{images/travailNeuflizeOBC/dashboard/kibanaGeneral_01.png}
    \caption{Extrait du dashboard n°1}
\end{sidewaysfigure}

\begin{sidewaysfigure}[ht]
	\centering
    \includegraphics[scale=0.5]{images/travailNeuflizeOBC/dashboard/kibanaGeneral_02.png}
    \caption{Extrait du dashboard n°2}
\end{sidewaysfigure}
\end{appendices}
%-----------------------------------------------------------------------------------------------------------------
%-----------------------------------------------------------------------------------------------------------------
\newpage
\newpage
% Page blanche

\end{document}
