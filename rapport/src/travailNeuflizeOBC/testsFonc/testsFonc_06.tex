	L'équipe avait émis le besoin de mettre en place des tests de charge, JMeter étant installé et y étant formé j'ai pris en charge la réalisation de ces tests. Ces derniers consistent à mesurer le temps de réponse d'un système lorsque celui-ci est soumis à des conditions particulières afin de vérifier s'il est capable de soutenir le traffic attendu. Les conditions regroupent différents paramètres comme le temps de réponse, le volume du traffic, le nombre de requêtes effectuées en parallèle, la configuration du matériel ou encore la stabilité des serveurs. Afin de réaliser de tels tests, il faut dans un premier temps configurer l'environnement sur lequel ils seront exécutés. Dans notre cas, nous avions à notre disposition un environnement de pré-production (RGB) dont les caractéristiques étaient identiques à celles de la production, ce qui était approprié pour réaliser ce genre de tests. De plus, nous avions aussi l'environnement HOMO3 avec des capacités inférieures, idéal pour exécuter les tests dans des conditions un peu plus extrêmes dans le but mesurer la robustesse des APIs. \\
	
	Après cela, j'ai pris soin de définir différents scénarios de tests dont l'objectif étaient de simuler l'activité utilisateur afin de se rapprocher le plus possible des cas d'utilisation réels. Par exemple, un des scénario classique pourrait décrire la procédure suivie pour effectuer une transaction bancaire d'un compte vers un autre. Pour cela, l'utilisateur pourrait se connecter à l'application mobile puis initier sa transaction en choisissant des comptes débiteur et créditeur et il enfin il la signerait numériquement. De plus, celui-ci pourrait d'abord consulter ses comptes avant de prendre la décision de réaliser cette transaction. Les scénarios sont représentés par des diagrammes de cas d'utilisation comme celui présent sur la figure \ref{scenarioTest}, décrivant la réalisation d'une transaction. \\

	On peut ainsi remarquer rapidement les différents services (listés en annexe \ref{a2}) qui seront appelés lors de l'exécution de ce scénario :
	\begin{itemize}
		\item VKB pour générer le clavier virtuel permettant l'authentification et la signature de la transaction
		\item tokenInfo pour générer le token d'authentification
		\item newTransfer pour initier la transaction
		\item accountToDebit pour sélectionner le compte débiteur
		\item addressBook pour sélectionner le compte créditeur
		\item signByVKB pour signer la transaction
		\item clientList pour lister tous les comptes
		\item accountOverview pour consulter un compte particulier \\
	\end{itemize}
	
	Une fois les services identifiés, il ne restait plus qu'à construire le plan de test JMeter permettant de réaliser le scénario. Pour cela, j'ai réemployé la structure mise en place précédemment pour les tests fonctionnels. Cependant, j'ai supprimé l'ensemble des assertions puisque ces dernières peuvent influer sur les temps de réponse calculés par JMeter et n'ont pas d'intérêt dans un test de charge. Il ne restait donc qu'à modifier les scripts Beanshell déjà écrits pour les adapter à la situation. La principale différence réside dans la configuration du moteur d'utilisateur. En effet, sur les tests fonctionnels j'avais placé un tel moteur à la racine des plans afin qu'il englobe tous les composants. Ainsi, pour les tests de charges il suffisait de modifier les paramètres du moteur à savoir le nombre de thread, d'itération et le temps de montée en charge pour adapter les conditions d'exécution des scénarios sans avoir à modifier la structure des plans.
	
	Ces tests de charges ont par la suite été exécutés dans différentes conditions ce qui a permis de déceler certaines anomalies. \hl{TODO : exemple du probleme voir abdel}