	Le dashboard étant maintenant terminé, une phase de recette a été organisée avec deux métiers ainsi que mon chef de projet afin que je puisse présenter le travail que j'avais effectué. Des extraits sont disponibles en annexe \ref{c2}. Tous les graphiques ont été passés en revue et les métiers ont pu vérifier que chacun des besoins que je m'étais engagé à satisfaire était effectivement présent sur ledit dashboard. Au terme de cette réunion, le dashboard a été approuvé malgré les quelques spécifications qui n'avaient pas pu être satisfaites et nous avons donc décidé de le mettre en place sur tous les environnements à savoir Homo3, RGB et la production. Nous avions déjà schématisé ces environnements figure \ref{environnement} mais ces derniers étaient incomplets. En effet, en réalité, chacun d'entre-eux possède différentes VM dont une dite de \textit{supervision} contenant les outils de monitoring et de supervision et une appelée \textit{microservice} sur laquelle se trouve le code source de l'API microservices. Les environnements RGB et de prod possèdent même deux VM microservices pour le load-balancing. La stack ELK a été installé sur supervision alors que FileBeat a été installé sur microservices (a et b) puisqu'il devait récupérer les logs générés par l'API. J'ai eu la chance de pouvoir procéder moi-même à l'installation de tous les éléments sur RGB. Après cela, je me suis occupé de la maintenance de ces outils afin de corriger les éventuels problèmes qui apparaissaient.\\
	
\begin{figure}[h!]
	\includegraphics[scale=0.45]{images/travailNeuflizeOBC/dashboard/elkDeploiement.png}
	\centering
	\caption{Architecture de la stack ELK sur tous les environnements}
	\label{elkDeploiement}
\end{figure}

	J'ai décrit jusqu'ici les principaux travaux que j'ai mené chez Neuflize OBC. Cependant, il reste de nombreuses tâches que j'ai réalisé comme la rédaction d'un guide complet concernant le projet destiné à l'équipe de maintenance qui s'occupera prochainement de nos API puisque celles-ci sont passées en production. J'ai aussi procédé à la réalisation de certaines tâches dites "architectes" via la mise en place de cache grâce à \textit{Redis} ou encore la configuration de RabbitMQ, le message broker, sur les différents environnements. Il s'agissait de petites tâches que j'ai pu mener conjointement avec l'un des architecte du projet, ce qui m'a permis de découvrir une facette de plus du projet (adminsitration via les consoles, TSE etc...) Nous allons maintenant nous intéresser plus en détails au travail que j'ai réalisé chez la Banque Privée 1818.