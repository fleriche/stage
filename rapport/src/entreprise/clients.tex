\subsection{Neuflize OBC}

\begin{figure}[h]
	\includegraphics[scale=0.3]{images/neuflizeOBCLogo.png}
	\centering
%	\caption{Logo Neuflize OBC}
	\label{neuflizeOBCLogo}
\end{figure}

\paragraph{}
La première mission sur laquelle j'ai été assigné a commencé le 13 février, d'abord à temps plein puis à mi-temps à partir du 29 mars (mercredi, jeudi) pour le compte du client Neuflize OBC. Ce dernier est une banque française privée, fruit de la fusion entre la banque \textit{Neuflize Schlumberger Mallet Demachy} (NSMD) et la banque \textit{Odier Bungener et Courvoisier} (OBC) en 2006. Neuflize OBC est devenue par la suite une filiale de la Banque Générale des Pays-Bas \textit{ABN AMRO} dont le capital est détenu à 100\% par l'état néerlandais. Mon stage s'est déroulé dans les locaux du siège social de Neuflize situé dans le 8ème arrondissement de Paris.

\subsection{BP1818}

\begin{figure}[h]
	\includegraphics[scale=0.3]{images/bp1818Logo.png}
	\centering
%	\caption{Logo BP1818}
	\label{bp1818Logo}
\end{figure}

\paragraph{}
La seconde mission sur laquelle j'ai été assigné a commencé le 29 mars à mi-temps (lundi, mardi et vendredi) pour le compte du client BP1818. Celui-ci est aussi une banque française privée, filiale de \textit{Natixis}, une banque créé en 2006 et fait partie du groupe \textit{Banque Populaire et Caisse d'Epargne} (BPCE) connu comme étant le deuxième acteur bancaire en France. Elle compte environ 500 collaborateurs et gère plus de 29 milliards d'euros à ce jour.

%Cependant, comme nous l'avons déjà dit plus haut, la société actuelle est marquée par de profondes mutations causées par l'explosion d'internet et des technologies du numérique. Ainsi, les usages et besoins clients ont grandement été impacté 