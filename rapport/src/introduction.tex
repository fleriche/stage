\paragraph{}
Le présent rapport détaille le travail qui s’inscrit dans le cadre de la réalisation d’un stage ingénieur que j'ai effectué en tant qu'élève ingénieur de l’Institut National des Sciences Appliquées de Rouen. Celui-ci s’est déroulé via la participation à deux missions distinctes pour le compte de l'ESN (Entreprise de Services du Numérique) \textit{Sopra Steria} du 13 février au 12 juillet 2017; suite à une convention tripartite signée entre le département Architecture des Systèmes d’Information de L’INSA, l’entreprise d’accueil et moi-même. Pendant ce stage, j’ai eu l’occasion de participer à plusieurs projets qui m’ont permis d’appréhender le métier d’ingénieur en informatique, d’acquérir de l’expérience ainsi que de m’épanouir aussi bien sur le plan personnel que professionnel.

\paragraph{}
La première mission à laquelle j'ai pris part s'est déroulée du 13 février 2017 jusqu'à la fin de mon stage, pour le compte du client \textit{Neuflize OBC}. La seconde mission a débuté le 29 mars 2017 pour la banque privée \textit{BP1818} et s'est déroulée en parallèle de la première jusqu'à la fin du stage. La répartition du temps de travail était la suivante :
\begin{itemize}
	\item Lundi, mardi et vendredi chez le client \textit{BP1818}
	\item Mercredi et jeudi chez le client \textit{Neuflize OBC}
\end{itemize}

\paragraph{}
Sopra Steria est une entreprise de services du numérique et l'un des leader européens dans la transformation numérique. Ainsi, l'objectif premier de mon stage a été d'accompagner certains des clients banques privées de Sopra Steria dans leur transformation digitale en travaillant à la fois sur la réalisation et l'industrialisation d'une application mobile à destination des clients de la banque Neuflize OBC ainsi que sur la mise en place d'une application web à destination des banquiers travaillant chez BP1818.

\paragraph{}
Dans un premier temps, je présenterai l'entreprise d'accueil et les clients pour lesquels j'ai travaillé, leurs domaines d’activités, leurs origines, leur personnel ainsi que leurs locaux.
Dans un second temps, je détaillerai de manière précise les sujets des missions qui m’ont été confiées au sein des équipes.
Enfin, je décrirai en profondeur le déroulement de mon stage ainsi que les différents travaux que j’ai accompli et les conditions dans lesquelles je les ai réalisé. 

\paragraph{}
Pour des raisons de clarté et de cohérence, ce rapport présentera deux parties distinctes concernant le déroulement du stage, chacune ayant pour objectif de décrire le travail réalisé chez les clients cités plus haut et suivant le même schéma.