	Le secteur bancaire est actuellement en effervescence, impacté par la venue de nombreux nouveaux acteurs qui viennent boulverser les règles établies. En effet, le progrès dans le domaine de l'informatique a permis de fournir des services inédits pouvant répondre aux attentes de la nouvelle génération de client marquée par l'essor des technologies du numérique. Ainsi, de nouveaux acteurs sont apparus, les banques en ligne, c'est-à-dire des banques uniquement disponibles sur internet. Ces dernières ne possèdent pas de locaux physiques et n'ont que très peu de personnel. Néanmoins, elles sont capables de permettre à leur clients de suivre l'état de leur compte bancaire ou patrimoine financier en temps réel. Il est possible de commander une carte bleue, un chéquier ou encore d'obtenir un rib sans avoir à se déplacer et bien d'autres services sont disponibles selon les banques et cela gratuitement. Les coûts de gestion des ressources humaines, d'organisation ou encore de matériels sont grandement réduit ce qui permet à ces nouvelles banques de pouvoir défier les grands groupes présents à l'échelle nationale. Ces derniers ont donc la nécessité de réagir afin de rester compétitif sur ce marché en pleine évolution. \\

	On parle ainsi de transformation digitale des banques pour désigner la transition marquée par le processus de dématérialisation de l'économie en faisant appel aux technologies modernes. Neuflize OBC, comme nous l'avons vu dans la partie précédente, est une banque privée. Cette dernière est actuellement en cours de transformation afin de pouvoir répondre aux besoins de ses clients. Un \href{https://www.neuflizeobc.net/portail/portail.jsp}{site web} a été créé permettant de proposer les services dits "banque au quotidien" qui regroupe les fonctionnalités classiques à savoir consultation de comptes, impression de rib ou encore réalisation d'une transaction bancaire. Cependant, les exigences sont toujours plus élevées, la génération actuelle étant toujours connectée via l'utilisation d'un smartphone, Neuflize s'est vu attribuer le besoin de produire une application mobile dans le but de permettre à ses clients de pouvoir accéder à leurs informations n'importe où et n'importe quand. \\

	L'équipe de développement constituée par Sopra Steria était en charge de la réalisation de la partie \textit{backend} de l'application, la partie \textit{frontend} ayant été déléguée à l'équipe qui a conçu le site web. Neuflize a décidé d'exposer des API dans le but de permettre la réalisation d'échanges au sein de son SI et vers l'extérieur. Ainsi, une architecture multicouches a été mise en place avec notamment :

\begin{itemize}
	\item Une couche d'API Management
	\item Une couche de micro services
	\item Une couche API Backend \\
\end{itemize} 

	La couche backend a pour objectif d'exposer des services unitaires développés par \textit{Elcimaï Financial Software} (EFS), un éditeur spécialisé dans la dématérialisation des flux financiers, créateur de la solution \textbf{WeBank}. Cette solution logicielle propose des services grandement sécurisés permettant de mettre en place de l'authentification via l'utilisation de tokens, de la gestion de portefeuilles titres ou encore la signature de numérique de transactions bancaires. Ces web services sont actuellement utilisés par le site internet de la banque et sont donc réemployés pour l'application mobile afin d'assurer une certaine cohérence. \\

	La couche micro services est au coeur du sujet de ce stage. En effet, j'ai intégré l'équipe de projet en charge du développement de cette couche. Cependant, la phase de développement touchant à sa fin, j'ai principalement participé à la phase d'industrialisation via la réalisation de certains travaux indépendants (automatisation de tests fonctionnels, tests de charges ou encore dashboards pour le client). C'est en majeure partie pour cette raison que j'ai été assigné sur un second projet dont le développement venait tout récemment de commencer.