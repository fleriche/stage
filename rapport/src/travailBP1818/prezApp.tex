	Banque Privée 1818 (nous abrégerons maintenant BP1818) a recours à l'une des Digital Factory de Sopra Steria dans le but de mettre en oeuvre sa stratégie digital et réaliser le projet \textit{Fronting Digital}. La Digital Factory est une plate-forme transversale permettant de réaliser des projets numériques et d'accompagner les métiers dans leur évolution. Celles-ci est constituée d'environ une vingtaine de personnes dont la première moitié est une équipe métier issue de la banque alors que la seconde est composée de profils IT issus de la DSI Natixis BP1818 et de l'équipe Sopra Steria. Au sein de cette factory, les mots d'ordre sont agilité et autonomie. En effet, l'objectif est d'amener les équipes à jouir d'une nouvelle forme de collaboration favorisant les liens entre les différentes parties prenantes et s'appuyant sur un management horizontal des équipes : tout le monde est impliqué dans le processus décisionnel et peut participer en apportant son point de vue ou ses idées. La démarche de la Digital Factory est d'assister le client dans sa transformation digitale en lui apportant des services de conseil et d'innovation mais aussi d'accompagnement UX/UI via la mise en place d'ateliers divers et en lui apportant l'expertise technique nécessaire pour mener à bien ses projets informatiques. \\
	
	BP1818 a le besoin de devenir une banque entièrement agile centrée autour de l'expérience client. En effet, elle souhaite pouvoir créer des espaces d'innovation afin de recevoir ses clients et démarrer de nouveaux projets toujours en suivant la philosophie agile. Les équipes au sein de ses projets, comme la notre, peuvent agir en autonomie favorisant ainsi la rapidité et la mise en place des décisions. Notre équipe se charge de la réalisation du Fronting Digital visant à développer une application web à destinations des banquiers et à accompgner ses derniers dans l'utilisation de ce nouvel outil. Pour cela, la banque mise sur une forte conduite du changement ainsi que sur l'utilisation de technologies récentes telles qu'Angular2, des éléments de la stack Netflix ou encore Couchbase une base de données NoSQL. Elle a fait le choix d'avoir aussi recours à des API externes comme Google pour gérer les données personnelles des clients ou SIREN pour les données entreprises. L'un des points crucial concernera l'intégration de ces nouveaux outils dans son SI. \\
	
	L'application web Fronting Digital permettra au banquier de faciliter leurs entrées en relation, les ouvertures de comptes ou encore les souscriptions d'assurance vie. Cette application est dotée d'un écran d'accueil proposant les fonctionnalités suivantes :
	
\begin{table}[h!]
	\center
	\begin{tabular}{| c | c |}
     \hline
     Fonctionnalité & Description \\ \hline
     Personnes physiques (PP) & Permet de saisir les données personnelles d'un client \\ & particulier ainsi que ses réponses aux questions de la MIF\\ \hline
     Personnes morales (PM) & Permet de saisir les données d'un client moral \\ & (entreprise, association, etc ...), de son représentant légal ainsi que ses réponses \\ & aux questions de la MIF\\ \hline
     Recherche client & Permet de rechercher un client PP ou PM enregistré \\ & en base ou dans le local storage \\ \hline
     Documentation & Fourni la documentation de l'application \\ \hline
     Dossiers en cours & Permet de retouver les dossiers en cours de création \\ & et d'accéder au suivi d'avancement d'un dossier\\ \hline
	\end{tabular}
	\caption{Fonctionnalités de l'application web Fronting Digital}
	\label{fonctionnalitesBP1818}
\end{table}

	La directive sur les Marchés d'Instruments Financiers (MIF) est une loi régissant l'organisation des marchés financiers en Europe. Lors d'une entrée en relation (PP ou PM), l'application propose différents onglets proposant des questions sur la MIF, les objectifs financiers ou encore les patrimoines et revenus qui seront posées au client. Une fois toutes les données et réponses aux questions saisies, il est possible de passer à l'étape de souscription. Ainsi, le client pourra choisir de souscrire à une assurance vie, d'ouvrir un compte titre, un compte de dépôt, choisir quel type de contrat il souhaite (capitalisation etc...). L'application permet aussi gérer l'import de fichiers, par exemple pour les pièces justificatives demandées lors d'une souscription comme dans n'importe quelle banque (carte d'identité, etc...). \\
	
	Dans cette application, lorsque le banquier commence la création d'un nouveau client, les informations sont sauvegardées dans le local storage de son navigateur. Le local storage permet aux applications web de stocker des données en plus grande quantité et de manière plus sécurisée qu'en ayant recours aux cookies. Lorsque le banquier a saisi suffisamment de données, le client se voit attribuer un id et est sauvegarder dans Couchbase, notre base de données dont nous parlerons plus en détail dans la partie suivante.