\subsection{Etude de l'user story}

	Le Fronting Digital est avant tout une application destiné aux banquiers afin de leur permettre de gérer plus efficacement et facilement leurs entrées en relation. Or, lors d'une ouverture de compte ou d'une souscription à une assurance vie il est obligatoire de fournir certains documents comme une pièce d'identité, un relevé d'identité bancaire ou encore un justificatif de domicile. Ici, ces documents sont appelés "pièces justificatives" et ont fait l'objet d'une user story sur laquelle j'ai travaillé de manière autonome, coinjointement avec un autre stagiaire. \\
	
	Le contenu de cette story concernait la mise en place d'un nouvel écran sur l'application depuis lequel il est possible d'uploader/downloader des pièces justificatives. Les pièces obligatoires qui doivent être fournies par le futur client de la banque dépendent de ses données personnelles. Par exemple, un client personne physique mineur ne fournira pas les même pièces qu'un client personne morale. Ainsi, le nouvel écran doit pré-afficher la liste des pièces obligatoires en la calculant au préalable à partir des données saisies par le banquier. La story mettait à notre disposition une matrice définissant l'ensemble des règles de gestion permettant de déterminer, à partir des données, si une pièce était obligatoire ou non. De plus, l'écran doit aussi proposer un menu déroulant permettant de choisir n'importe quel type de pièce afin de l'ajouter manuellement à la liste. Les pièces ajoutées de cette manière sont facultatives et doivent être affichées d'une couleur différente. Il est possible de retrouver en annexe \ref{d1} une maquette illustrant ce nouvel écran. \\

	Comme nous l'avons expliqué dans la partie \ref{deroulementSprint}, la première étape pour l'équipe de développement consiste à prendre connaissance de l'user story afin de la découper en tâches unitaires. Ici, nous avons fait le choix de créer 5 tâches qui sont les suivantes :
	
	\subsubsection{Composant pièce justificative}
	Le composant "pièce justificative" est un composant Angular représentant la brique qui contient tous les éléments et options d'une pièce justificative à savoir :
	\begin{itemize}
		\item Le fichier
		\item La date de fin de validité
		\item L'accord dérogatoire
		\item La date d'upload du fichier
		\item La possibilité de supprimer la pièce
		\item La possibilité de visualiser la pièce (ou la télécharger si elle ne peut être visualiser sur le navigateur)
	\end{itemize}
	Cette brique est représentée par un rectangle rouge sur l'annexe \ref{d1}
	
	\subsubsection{Composant écran pièce justificative}
	Ce composant est l'écran sur lequel sont affichées les pièces justificatives sous la forme d'une liste dont les éléments sont les composants décrits précédemment. Celui-ci pré-remplie la liste avec les pièces définies comme étant obligatoires. Ainsi, le banquier aura un rapide aperçu des pièces qu'il doit demander à son client et n'aura qu'à uploader les fichiers pour compléter la liste. De plus, un menu déroulant doit être présent, permettant de rajouter des pièces manuellement à la liste, dites facultatives qui apparaissent d'une couleur différente. En outre, le service d'appel front devra être créé afin de pouvoir communiquer avec l'API et consommer les web services. En outre, l'écran doit être accessible depuis la barre de navigation principale de l'application et un compteur indiquant le nombre de pièces obligatoires fournies/totales (par exemple 2/5) doit figurer au-dessus de la barre.
	
	\subsubsection{Référentiel}
	Dans notre application apparait un certain nombre de données métiers constantes comme la liste des pays pour un client, la liste des départements pour la France ou encore la liste des formes juridiques pour une entreprise. Pour toutes ces listes de données un endpoint est ajouté et un service appelé \textit{référentiel} est créé. Ainsi, il est possible de les obtenir de manière simple depuis le front en consommant le service adéquat. Si les métiers ont le besoin de modifier les valeurs, il suffit d'apporter une simple modification rapide et cela permet de garder un code structuré et clair. Dans notre cas, un référentiel devait être créé afin de référencer tous les types de pièces justificatives prises en charge par la banque (carte d'identité, justificatif de domicile, etc... pour un total de 41 différentes).	
	
	\subsubsection{Backend}
	Comme son nom l'indique, cette tâche consiste à développer les services backend destinés à alimenter notre frontend. Cela implique la création du contrôleur et de tous les endpoints, le service dédié ainsi que le modèle permettant de gérer les pièces justificatives. De plus, cela inclut le paramétrage pour la limite de taille des fichiers à uploader ainsi que la validation des données reçues.
	
	\subsubsection{Moteur de calcul}
	Nous avons dit plus haut que l'écran des pièces dévait pré-afficher la liste des pièces obligatoires. Or, le fait qu'une pièce soit obligatoire ou non dépend des données personnelles du client de la banque. Ainsi, le moteur de calcul doit prendre en compte l'ensemble des règles de gestion métiers ainsi que les données entrées jusqu'à présent par le banquier dans l'application afin de définir la liste des pièces obligatoires puis la transmettre au composant écran. \\
	
	Comme je l'ai dit plus haut, j'ai travaillé sur cette story avec un autre stagiaire. Ainsi, nous nous sommes séparé les tâches avant de commencer c'est pourquoi je me suis plutôt occupé de la partie backend et lui de la partie frontend. Ce découpage permettait d'avancer en parallèle sans problèmes. \\
	
\subsection{Sprint planning}
	Après avoir étudier l'user story et définie les différentes tâches à réaliser, une réunion de type sprint planning a eu lieu afin de définir la charge de travail pour effectuer chacune d'entre elles. Au cours de cette réunion nous avons voté et la charge a été réparti de la manière suivante :
	
\begin{table}[h!]
	\center
	\begin{tabular}{| c | c |}
     \hline
     Tâche & Charge (jour) \\ \hline
     Composant pièce justificative & ?\\ \hline
     Composant écran justificatifs & ?\\ \hline
     Backend & ?\\ \hline
     Référentiel & ?\\ \hline
     Moteur de calcul & ?\\
     \hline
	\end{tabular}
	\caption{Sprint planning pièces justificatives}
	\label{sprintPlanningPJ}
\end{table}

	Une fois le sprint planning terminé nous avons pu passer à la conception de nos tâches respectives ainsi qu'à la phase de développement.

\subsection{Conception}

	\subsubsection{Endpoints - couche controller}

	Dans un premier temps, j'ai commencé par définir les différents cas d'utilisation possible d'un banquier qui utiliserait l'écran des pièces justificatives. Il est possible d'observer ces cas sur le diagramme figure \ref{useCasePJ}.

\begin{figure}[h!]
	\includegraphics[scale=0.50]{images/travailBP1818/piecesJustif/useCasePJ.png}
	\centering
	\caption{Cas d'utilisation}
	\label{useCasePJ}
\end{figure}

	Pour chacun de ces derniers j'ai décidé de mettre en place un service permettant de répondre au besoin qu'il définissait. J'ai ainsi pu procéder à la création du contrôleur ainsi que des endpoints permettant d'exposer les services répondant aux cas d'utilisation. Pour cela, il a fallu déterminer les méthodes HTTP et l'url associée que notre client pourra intéroger en essayant de rester RESTful. Le tableau suivant décrit les endpoints ainsi créés :
	
\begin{table}[h!]
	\center
	\begin{tabular}{| c | c | c |}
     \hline
     Cas d'utilisation & Méthode HTTP & URL \\ \hline
     Obtenir pièces justificatives & GET & /justificatifs/{idForm}\\ \hline
     Créer pièce justificative & POST & /justificatifs/{idForm}/justificatif\\ \hline
     Modifier pièce justificatives & PUT & /justificatifs/{idForm}/justificatif\\ \hline
     Supprimer pièce justificative & DELETE & /justificatifs/{idForm}/justificatif\\ \hline
     Downloader fichier & GET & /{idForm}/fichier\\ \hline
     Uploader fichier & POST & /{idForm}/fichier\\ \hline
     Supprimer fichier & DELETE & /{idForm}/fichier\\ \hline
	\end{tabular}
	\caption{Sprint planning pièces justificatives}
	\label{sprintPlanningPJ}
\end{table}

	\subsubsection{Modèle - couche domain}
	
	Comme nous l'avons vu dans la partie \ref{archiBP1818}, nous avons recours à Couchbase pour notre base de données de type NoSQL. Ainsi, toutes les données sont enregistrées sous la forme de documents au format JSON. Dans notre cas, nous avons généralement un document par formulaire dont le nom est constitué de l'id du client et d'un préfixe permettant d'identifier rapidement le formulaire. Par exemple, nous avons un écran sur lequel il est possible de spécifier les données personnelles du client. Le formulaire est sauvegardé dans un document portant le nom \textit{TIERS\_10001} pour le client d'id 10001. Nous avons un écran déroulant une série de questions sur les patrimoines et revenus du client. Le formulaire regroupe les réponses à toutes les questions et est sauvegardé dans le document \textit{PAT\_10001}, toujours pour le client d'id 10001. Ainsi, dans le cas des pièces justificatives j'ai décidé de sauvegarder l'ensemble du formulaire contenant les pièces dans un nouveau document dont le nom serait \textit{JUST\_ID}. \\
	
	Une fois cela définit, il fallait déterminer quelles données devaient être persistées ainsi que leur structure. J'ai donc commencé par définir le modèle qui serait utilisé en étudiant plus en profondeur l'user story afin de relever toutes les informations à persister. Ce modèle peut être représenter par le diagramme de classe figure \ref{modelePJ}.	
	
\begin{figure}[h!]
	\includegraphics[scale=0.7]{images/travailBP1818/piecesJustif/modelePJ.png}
	\centering
	\caption{Modèle des pièces justificatives}
	\label{modelePJ}
\end{figure}

	La classe \textit{Fichier} représente un fichier uploadé par le banquier et contient ses informations comme son nom, l'url indiquant à quel endroit est enregistré le fichier, la date à laquelle il a été enregistré et son content-type. Celui-ci est une en-tête permettant d'indiquer le type MIME d'une ressource, nécessaire dans notre cas pour visualiser le fichier dans le navigateur. \\
	
	Ensuite, la classe \textit{Justificatif} représente une pièce justificative et toutes les informations qui lui sont associées comme le type ou la date de validité. Elle contient aussi l'id du tiers ("idTiers") à qui elle appartient afin de pouvoir l'identifier ainsi que son identité ("identite") afin de pouvoir afficher le nom de son propriétaire côté frontend (par exemple carte d'identité de M. Jean Dupont). En outre, cette classe contient une liste de fichier ("fichiers") contenant tous les fichiers uploadés pour la pièce justificative (par exemple pour une carte d'identité il peut y avoir deux fichiers : une image pour le recto et une pour le verso). \\
	
	Enfin, la classe \textit{Justificatifs} contenant l'id du document au format JUST\_ID ainsi qu'une liste de pièces justificatives ("justificatifs"). \\
	
	J'ai pu implémenter ce modèle en Java en créant des Beans côté backend. Nous utilisons actuellement Maven comme outil de gestion et d'automatisation de production de logiciel. Nous avons configuré un plugin Maven, nommé \textit{Typescript Generator}, permettant de générer le modèle côté frontend. En effet, une fois le modèle créé côté backend, il suffit de relancer un build Maven afin que ce plugin génère le code Typescript à partir du code Java. Un fichier est créé contenant toutes les interfaces typescript obtenues à partir des Beans java. Cela nous permet d'assurer la cohérence entre nos modèles back et front et nous permet de gagner beaucoup de temps en nous évitant de réécrire une seconde fois le modèle.
	  
	
	\subsubsection{Communication avec Couchbase - couche repository}
	
	Le modèle étant défini, il fallait maintenant mettre en place la communication avec Couchbase afin de persister les données ou les récupérer. Pour cela, j'ai créé un repository pour les pièces justificatives. Comme nous l'avons vu plus haut, il s'agit d'une interface facilitant grandement l'utilisation de requêtes vers notre base de données. Le schéma figure \ref{repository} illustre le fonctionnement du repository des pièces justificatives. \\
	
	J'ai donc créé JustificatifsRepository et fait en sorte que la classe \textit{Justificatifs} implémente l'interface IdentifiableDocument. Pour cette user story aucune autre requête que celles fournies était nécessaire donc aucun développement supplémentaire était requis. \\

\begin{figure}[h!]
	\includegraphics[scale=0.55]{images/travailBP1818/piecesJustif/repositoryPJ.png}
	\centering
	\caption{Repository pièces justificatives}
	\label{repositoryPJ}
\end{figure}
		
	\subsubsection{Service - couche service}
	
	A ce stade, les cas d'utilisation étaient connus, le modèle défini, la communication avec Couchbase établie et les endpoints exposant les services prêts. La prochaine étape consistait à effectivement développer les services permettant de répondre aux différents besoins identifiés sur le diagramme figure \ref{useCasePJ}.
	
\begin{table}[h!]
	\center
	\begin{tabular}{| c | c |}
     \hline
     Services & Description \\ \hline
     findJustificatifs & Permet de retourner le formulaire JUST\_ID contenant toutes les pièces \\ \hline
     createPiece & Permet de créer pièce justificative en l'ajoutant au formulaire \\ \hline
     updatePiece & Permet de mettre à jour une pièce justificative du formulaire \\ \hline
     deletePiece & Permet de supprimer une pièce justificative en la supprimant du formulaire \\ \hline
     uploadFile & Permet d'uploader un fichier \\ \hline
     downloadFile & Permet de downloader un fichier \\ \hline
     deleteFile & Permet de supprimer un fichier\\ \hline
	\end{tabular}
	\caption{Services pièces justificatives}
	\label{servicesPJ}
\end{table}

\begin{figure}[h!]
	\includegraphics[scale=0.55]{images/travailBP1818/piecesJustif/seqSave.png}
	\centering
	\caption{Sauvegarde des pièces justificatives}
	\label{seqSave}
\end{figure}

\begin{figure}[h!]
	\includegraphics[scale=0.55]{images/travailBP1818/piecesJustif/seqGet.png}
	\centering
	\caption{Affichage des pièces justificatives}
	\label{seqGet}
\end{figure}

\subsection{Résultats et perspectives}

\begin{table}[h!]
	\center
	\begin{tabular}{| c | c | c |}
     \hline
     Tâche & Charge prévue (jour) & Charge réelle (jour) \\ \hline
     Composant pièce justificative & ? & ?\\ \hline
     Composant écran justificatifs & ? & ?\\ \hline
     Backend & ? & ?\\ \hline
     Référentiel & ? & ?\\ \hline
     Moteur de calcul & ? & ?\\
     \hline
	\end{tabular}
	\caption{Rétrospective pièces justificatives}
	\label{retroPJ}
\end{table}
	
	
	\newpage
\begin{itemize}
	\item prendre connaissance user story + besoin client OK
	\item découpage en tâche sur github OK
	\item sprint planning (tableau avec les temps) OK
	\item définition des endpoints OK
	\item définition du modèle OK
	\item référentiel OK
	\item service
	\item moteur de scoring front (règles de gestion)
	\item TU
	\item pb avec swagger
	\item resultat obtenu (retard tableau comparatif + pourquoi + mesure)
\end{itemize}