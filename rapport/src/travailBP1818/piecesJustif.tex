	Le Fronting Digital est avant tout une application destiné aux banquiers afin de leur permettre de gérer plus efficacement et facilement leurs entrées en relation. Or, lors d'une ouverture de compte ou d'une souscription à une assurance vie il est obligatoire de fournir certains documents comme une pièce d'identité, un relevé d'identité bancaire ou encore un justificatif de domicile. Ici, ces documents sont appelés "pièces justificatives" et ont fait l'objet d'une user story sur laquelle j'ai travaillé de manière autonome, coinjointement avec un autre stagiaire. \\
	
	Le contenu de cette story concernait la mise en place d'un nouvel écran sur l'application depuis lequel il est possible d'uploader/downloader des pièces justificatives. Les pièces obligatoires qui doivent être fournies par le futur client de la banque dépendent de ses données personnelles. Par exemple, un client personne physique mineur ne fournira pas les même pièces qu'un client personne morale. Ainsi, le nouvel écran doit pré-afficher la liste des pièces obligatoires en la calculant au préalable à partir des données saisies par le banquier. La story mettait à notre disposition une matrice définissant l'ensemble des règles de gestion permettant de déterminer, à partir des données, si une pièce était obligatoire ou non. De plus, l'écran doit aussi proposer un menu déroulant permettant de choisir n'importe quel type de pièce afin de l'ajouter manuellement à la liste. Les pièces ajoutées de cette manière sont facultatives et doivent être affichées d'une couleur différente. Il est possible de retrouver en annexe \ref{d1} une maquette illustrant ce nouvel écran. \\

	Comme nous l'avons expliqué dans la partie \ref{deroulementSprint}, la première étape pour l'équipe de développement consiste à prendre connaissance de l'user story afin de la découper en tâches unitaires. Ici, nous avons fait le choix de créer 5 tâches qui sont les suivantes :
	
	\subsubsection{Composant pièce justificative}
		
	
	\subsubsection{Ecran pièce justificative}
	\subsubsection{Référentiel}
	\subsubsection{Backend}
	\subsubsection{Moteur de calcul}
	
	\newpage
\begin{itemize}
	\item prendre connaissance user story + besoin client
	\item découpage en tâche sur github
	\item sprint planning (tableau avec les temps)
	\item définition du modèle
	\item définition des endpoints
	\item référentiel
	\item pb avec swagger
	\item moteur de scoring front (règles de gestion)
	\item TU
	\item resultat obtenu (retard tableau comparatif + pourquoi + mesure)
\end{itemize}