	Sur ce projet j'ai été intégré dans l'équipe de développement du Fronting Digital. Ainsi, contrairement au projet mené chez Neuflize OBC, je n'ai pas réalisé différents travaux indépendants mais j'ai directement participé au développement de l'application. Nous avons travaillé en suivant les concepts des méthodes agiles \cite{bib_agile} dont je vais parler plus en détail ci-dessous. La description de l'ensemble des travaux de développement que j'ai effectués serait fastidieuse et répétitive c'est pourquoi je vais d'abord expliciter le déroulement des sprints ainsi que les manières dont nous avons réalisé la conception des fonctionnalités puis je présenterai les tâches les plus représentatives que j'ai eu l'occasion d'effectuer en autonomie.

\subsection{Méthodes agiles}
	Comme nous l'avons dit plus haut, nous avons eu recours à un courant agile \cite{bib_agile2} pour la réalisation de ce projet. Il s'agit d'une approche de gestion de projet qui vient s'opposer aux méthodes plus traditionnelles et séquentielles comme le \textit{cycle en V}. On établit en début de projet l'ensemble des relations commerciales et des obligations de chaque partie. L'intégralité du produit doit être spécifiée et planifiée dans les détails ce qui aboutit à la création d'un cahier des charges très conséquent qui sera par la suite confié à l'équipe de développement. Après cela, les développeurs s'isolent sur une très grande période afin de réaliser le produit. Le client doit avoir pensé à absolument tout, de la disposition des pages de son application à la couleur des boutons des formulaires. Les changements nécessitent des avenants et sont réalisés souvent trop tard ce qui est contre-productif. De plus, tous les imprévus rencontrés lors de la phase de développement ont tendance à rendre la planification de départ obsolète. \\
	
	Les méthodes agiles sont centrées sur la satisfaction du besoin client plutôt que sur la conformité d'un contrat et favorise un raisonnement "produit" plutôt qu'un raisonnement "projet". L'objectif est de mettre en place des cycles courts nommés itérations ou sprints et de découper le projet en petits blocs. Le demandeur commence par exprimer ses exigences qu'il soumet aux développeurs en communicant directement avec eux, ce qui favorise l'entente et les relations et permet de gagner du temps. L'équipe choisit alors une partie des exigences ainsi qu'une courte période de temps pour effectuer les phases de spécification, conception, developpement et tests, au terme de laquelle le produit est montré au client. Celui-ci peut alors émettre des retours offrant de la \textbf{visibilité} qui seront pris en compte par l'équipe de développement. Les allers/retours entre le client et l'équipe de réalisation sont beaucoup plus courts avec à la clé gain en efficacité et productivité. Ainsi, il est possible de s'\textbf{adapter} aux souhaits du client, remplacer une fonctionnalité par une autre, modifier des styles, déplacer une page etc... et de réduire considérablement l'effet tunnel des approches classiques. En effet, il est possible d'éviter le surplus, les fonctionnalités qui ne seront finalement pas utilisées, de mettre en place les bonnes idées apparues pendant le projet et donc de véritablement enrichir le produit en mettant en avant la production de valeur pour l'utilisateur final (on parle de \textbf{business value}). Les \textbf{risques} repérés peuvent être traités beaucoup plus tôt en travaillant directement avec l'utilisateur final. \\

	Dans notre cas, nous utilisons la méthodologie \textit{Scrum} \cite{bib_agile3}. Il s'agit de l'une des méthodes agiles les plus utilisée dont nous allons maintenant voir le détail.

\subsection{Déroulement des sprints}
\label{deroulementSprint}

	Scrum est une méthode de gestion de projet très répandue et prisée des adeptes de la philosophie Agile. Bien qu'elle soit plutôt simple à comprendre, elle reste néanmoins difficile à mettre en place et à maîtriser dans un projet. \\
	
	\subsubsection{Sprints}
	Nous avons vu plus haut que les méthodes Agiles préconisent de découper la réalisation du projet en un ensemble d'itérations courtes. Ici, ces dernières sont nommées \textbf{sprints} qui sont des périodes de deux semaines (ou parfois trois semaines lorsque des conditions particulières l'imposent comme une diminution des ressources à cause des vacances).
	
	\subsubsection{Le Product Backlog}
	Le product backlog est un référentiel contenant les exigences fonctionnelles et non fonctionnelles initiales définies par le client. Celui-ci est amené à évoluer au cours de l'avancée du projet.
	
	\subsubsection{User Story}
	Une user story, ou récit utilisateur, est la description d'une fonctionnalité métier à valeur business. Dans notre cas, il s'agit d'un fichier excel décrivant une fonctionnalité du point de vue utilisateur (\textit{En tant que..., je clique sur … dans le but de …}). Elle présente toutes les règles de gestions, prévoit tous les cas d'utilisation et contient aussi des maquettes afin que les développeurs puissent construire l'IHM dans les meilleures conditions possibles.
	
	\subsubsection{Déroulement d'un sprint}
	L'équipe Sopra Steria se décompose en une équipe fonctionnelle, de pilotage ainsi que de développement. Lors d'un sprint, l'équipe fonctionnelle collabore avec les métiers BP1818 et le product owner afin de pouvoir constituer un backlog suffisamment complet et ordonnancé pour plannifier le prochain sprint. Pour cela des \textbf{ateliers} sont régulièrement organisés afin de faire constamment évoluer le product backlog. Après cela, l'équipe fonctionnelle est en charge de rédiger les \textbf{User Stories}. Une user story représente une exigence ou une fonctionnalité souhaitée par le client et décrit celle-ci de manière détaillée en précisant le comportement attendu, des maquettes pour l'IHM etc... : \textit{Lorsque je vais sur la page ... et que je clique sur le bouton ... l'action ... se déclenchera}. \\
	
	Une fois ces dernières validées, elles sont transmises à l'équipe de développement afin qu'elle puisse les étudier. Dans un premier temps, l’ensemble de l’user story est découpée en tâches unitaires pouvant être réalisées par une unique ressource. Par la suite, une réunion de type \textbf{Sprint Planning} est organisée avec l'ensemble de l'équipe Sopra Steria dans l’optique de valider le contenu du sprint. Lors de ce sprint planning nous présentons l’ensemble des tâches découpées précédemment à l’équipe et exposons brièvement la manière dont nous les développerons afin que chacun puisse s’imprégner du travail qui sera réalisé et que les dernières modifications puissent être apportées. Ensuite, pour chacune des tâches ainsi explicitées, une estimation de charge est réalisée via la méthode du \textbf{Poker Planning}. Il s’agit d’une pratique permettant d’estimer la complexité des fonctionnalités à développer en faisant interagir de manière ludique l’ensemble des parties prenantes. En effet, chaque participant possède des cartes indiquant différents niveaux de complexité (les niveaux utilisent les nombres de la suite de Fibonacci). Lorsqu’une des tâches a été présentée par les développeurs, les participants choisissent la carte correspondant à leur estimation personnelle de la complexité et la retournent tous en même temps. La discussion reprend tant qu’il n’y a pas l’unanimité sur la valeur de la complexité. Dans notre cas, un point de complexité correspond à 0.75 jour de travail pour une ressource. Par exemple, si une tâche est validée à 5 points de complexité alors cette dernière devra être réalisée en 3.75 jours. Une fois les estimations terminées, certaines tâches sont choisies en fonction de leur priorité définie avec le product owner et du temps disponible pour le sprint et constituront le \textbf{Sprint Backlog}.\\
	
	A l'issue de cette réunion la phase de développement peut commencer. Elle inclut la conception détaillée des différentes tâches, le développement des fonctionnalités et des tests unitaires. Lorsque le développement est terminé le produit est livré et part en phase de qualification. Cette phase est menée par l'équipe fonctionnelle qui est en charge d'effectuer une batterie de test afin de vérifier la conformité du produit à la user story. Tous les bugs relevés sont consignés sur Jira afin que les développeurs puissent les corriger. Cette phase est en quelque sorte une phase de pré-recette nous permettant de nous affranchir d'un maximum d'anomalies avant de livrer l'application au client.

\begin{figure}[h!]
	\includegraphics[scale=0.7]{images/travailBP1818/scrum.jpg}
	\centering
	\caption{Méthode Scrum}
	\label{scrum}
\end{figure}

	Une réunion nommée \textit{Daily meeting} a lieu chaque jour dans le but de vérifier que les objectifs seront bien atteints. Le but de cette réunion est de répondre aux trois questions suivantes : 
	\begin{itemize}
		\item Qu'est-ce que j'ai fait ?
		\item Qu'est-ce que je vais faire ?
		\item Quels sont les problèmes que je rencontre ?
	\end{itemize}
	Elle regroupe l'ensemble des équipes intervenant sur le projet à savoir Sopra Steria, la DSI ainsi que les métiers et dure 15 minutes. Un second daily meeting a lieu juste avant, lui aussi de 15 minutes mais seulement avec l'équipe Sopra Steria. \\

	Une fois par semaine a lieu une réunion nommée \textit{V1} ne concernant elle aussi que l'équipe Sopra Steria et dans laquelle un bilan est effectué concernant l'avancement du projet, les problèmes majeurs rencontrés, les retards possibles, les mesures à prendre etc... Cette dernière est présidée par notre directeur de projet. \\
	
	Personnellement, je trouve que l'ensemble de ces réunions permet de soulever les problèmes et favorise la collaboration en permettant à tout le monde d'invervenir et de proposer ses idées. Cependant, leur trop grand nombre implique une grande perte de temps, temps souvent précieux pour l'équipe de développement. Durant le daily meeting de Sopra Steria nous parlons en détail du développement et de l'application. Or, nous ne pouvons aborder ces points techniques durant celui avec les métiers. Lors du daily métier, il est fréquent que seul les points métiers soient évoqués, points dont l'équipe de développement ne saisit pas forcément le sens. De plus, il n'est pas rare que le temps de 15 minutes soit dépassé. \\
	
	Concernant les sprint planning, il s'agit d'une réunion très intéressante permettant d'estimer la charge en faisant, là aussi, intervenir tout le monde. Néanmoins, il est fréquent que le temps alloué à cette dernière soit dépassé, voire parfois nettement dépassé. En effet, le temps peut parfois atteindre jusqu'à 3h. Cela est dû au fait que nous n'avons jamais de \textit{time keeper} ou gardien du temps afin d'aiguiller la réunion. Parfois d'autres points sont abordés lors de la réunion et certaines digressions font perdre du temps. Le temps total passé en réunion est souvent supérieur à ce qu'il aurait dû être et manque à la phase de développement. \\
	
	Enfin, une fois la phase de qualification terminée et le projet livré au client, une réunion \textit{Sprint review} est organisée avec l'ensemble de l'équipe de projet et est animée par l'équipe fonctionnelle Sopra Steria. L'objectif est de présenter ce qui a été développé au cours du sprint au client et noter les différentes remarques que celui-ci peut être amené à faire. Ces réunions sont un moyen très efficace de communiquer avec le client et nous permettent de prendre en compte ses remarques et idées sur le projet. Elle s'inscrit bien dans le courant agile en favorisant les intéractions entre les différents collaborateurs, en permettant à chacun de s'intégrer au sein du projet et de se sentir acteur dans le cadre du Fronting Digital. Il s'agit là d'un point que j'ai particulièrement apprécié au sein de l'équipe, balayant les préjugés classiques sur les gros projets tels que la phase d'isolement des développeurs après l'établissement des spécifications. Ici la communication est au coeur de la réussite du projet et permet d'assurer un résultat conforme aux besoins et aux attentes du client, même si ces dernières évoluent au cours du temps. \\
	
	Néanmoins, malgré une bonne communication, il reste parfois certains désaccords avec le client. Le client souhaitait ne pas avoir de bouton de sauvegarde sur les différents formulaires présents dans ces onglets. Cette absence de bouton implique un grand nombre de requêtes vers notre API afin de sauvegarder sans cesse les informations entrées par le banquier. Je trouve que cette absence n'apporte pas de plus value à l'application et est une erreur aussi bien au niveau expérience utilisateur (lors d'un premier essai de l'application tout le monde a tendance à chercher le bouton de sauvegarde sur les formulaire) qu'au niveau performance. Il s'agit d'un des exemples dans mon cas de désaccord avec le client qui aurait peut-être nécessité plus ample discussion.