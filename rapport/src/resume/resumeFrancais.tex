\fancyhead[LE]{
	\begin{picture}(0,0) 
	\put(-30,-8){}
	\end{picture}
}
\fancyhead[LO]{
	\begin{picture}(0,0) 
	\put(-8,-8){}
	\end{picture}
}

	Le présent document est un rapport de stage de fin d'étude décrivant mon travail dans le domaine du Digital Banking. L'industrie bancaire a été révolutionnée par les nouvelles technologies de l'information et connait maintenant un bouleversement. Beaucoup de nouveaux compétiteurs peuvent maintenant s'insérer sur le marché et défier les grands groupes en prenant avantage de l'innovation digitale. Ce sont des banques agiles, centrées sur l'expérience client utilisant les technologies digitales afin de fournir des services bancaires grandement sécurisé à leurs clients. Dans ce rapport je vais décrire le travail que j'ai effectué sur la transformation digitale de deux banques : Neuflize OBC et Banque Privée 1818. Ces banques ont décidé d'avoir recours aux services de Sopra Steria pour être accompagnés dans leur processus de digitalisation. L'objectif est de leur fournir tous les éléments bancaires dont ils ont besoin en se concentrant sur l'expérience client et cela indépendamment de leur évolution. Qu'ils soient nouveaux sur le marché ou déjà établis nous devons leur livrer des services flexibles pouvant s'adapter à leurs besoins à mesure qu'ils évoluent. Dans mon cas, j'ai travaillé, d'une part, sur le développement du backend d'une application mobile pour Neuflize OBC. Il s'agissait d'une API REST avec une architecture microservices développée à l'aide de la stack open source de Netflix. D'autre part, j'ai travaillé sur le développement d'une application web pour Banque Privée 1818. Son backend est une API REST construite à l'aide du framework Spring Boot et son frontend un client Angular4, le tout développé en appliquant un paradigme de programmation réactive.