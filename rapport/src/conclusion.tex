\newenvironment{changemargin}[2]{%
\begin{list}{}{%
\setlength{\topsep}{0pt}%
\setlength{\leftmargin}{#1}%
\setlength{\rightmargin}{#2}%
\setlength{\listparindent}{\parindent}%
\setlength{\itemindent}{\parindent}%
\setlength{\parsep}{\parskip}%
}%
\item[]}{\end{list}}

\begin{changemargin}{-1cm}{-1cm}

	Ce stage a été, pour moi, extrêmement enrichissant de par la richesse des tâches que j'ai pu effectuer. En effet, j'ai eu la chance de pouvoir travailler à mi-temps sur deux projets importants de l'agence 512, plus grosse agence sur secteur bancaire de Sopra Steria. Ces projets digitaux sont actuellement des modèles ainsi que des portes d'entrées vers de nouveaux projets et clients voulant franchir le pas de la transformation numérique. Sur ces projets j'ai pu à chaque fois travailler au sein d'équipes de développement dynamiques intégrées dans leur environnement et centrée autour de la collaboration pour la réussite d'un projet digital dans son ensemble et non pas pour la simple réalisation d'un produit à vendre. Aussi, de par l'utilisation des méthodes agiles et des relations ainsi créées j'ai pu sortir du cadre développeur classique afin de m'entretenir directement avec le client, que ce soit pour revoir des spécifications et le conseiller comme sur le sujet des dashboards, pour des recettes ou encore pour que celui-ci m'éclaircisse son point de vue et son besoin. Cela permet d'avoir une idée nettement plus claire de ses attentes mais aussi des contraintes auxquelles il doit faire face et donc d'être force de proposition dans le but de lui venir en aide. Il s'agit là d'une véritable plus-value pour mon insertion professionelle puisque j'ai eu la chance de travailler et d'intéragir non seulement avec des développeurs mais aussi avec des architectes, des équipes fonctionnelles, de pilotage, de conduite du changement, des commerciaux et bien sûr des équipes métiers. \\
	
	Durant mon stage j'ai eu la chance d'être intégré à la "communauté architecture" de la division banque et finance de Sopra Steria dans laquelle une veille technologique est assurée et des brown bag lunch (BBL) sont organisés. Il s'agit de présentations sur des sujets technologiques variés effectuées sur l'heure du repas par des développeurs ou des architectes membres de la communauté. Cela apporte une fois encore un côté humain favorisant les relations tout en permettant de rester centré autour des métiers de l'information. Les architectes essaient de présenter les dernières technologies et n'hésitent pas à les employer dans leurs projets actuels. Par exemple, chez Neuflize OBC, une architecture microservices a été mise en place, une première dans la division bancaire Sopra Steria. Le succès de cette dernière a pu être transmis par le biais de la communauté et une étude est menée en ce moment par des développeurs de chez BForBank pour l'utiliser sur l'un de leur projet. De plus, les outils utilisés sont ceux de la stack Netflix, récents et open-source. Du côté de BP1818, nous avons recours à Angular2 pour construire notre frontend et nous avons même migré sur Angular4 sorti en version stable au mois de mars 2017, puis Angular5 en juillet. \\
	
	Un autre point fort que j'ai pu relever durant ce stage concerne la richesse des travaux que j'ai pu réaliser. Du côté de BP1818 j'ai pu effectuer du développement aussi bien sur la partir backend que sur la partir frontend. Une certaine liberté m'a été laissé puisque j'ai souvent pu réaliser mes tâches de manière autonome tant au niveau conception que développement bien qu'étant sous la supervision de mon tuteur en cas de nécessité. Chez Neuflize OBC, je n'ai effectué que peu de développement mais j'ai pu m'intéresser à la partie tests fonctionnels de par leur réalisation et la mise en place d'outils pour les automatiser et faire gagner du temps à l'équipe. J'ai aussi pu mettre en place du monitoring applicatif, sujet durant lequel je pouvais moi-même organiser des points avec les métiers. De plus, une fois mes sujets terminés et avant mon passage à temps plein chez BP1818 l'un des architectes m'a permi d'effectuer des tâches sur les environnements de pré-production et de production comme l'installation de la stack ELK, la configuration RabbitMQ ou encore la résolution de certains problèmes d'infrastructure. Cela m'a apporté de nombreuses connaissances, une importante stack et m'a encore une fois permi de sortir de la seule vision "développeur". \\
	
	Fort de cette expérience, j'ai accepté l'offre d'emploi que Sopra Steria m'a proposé, les deux projets ayant comblés mes attentes et m'ayant permi de grandement m'épanouir. Le projet chez Neuflize OBC étant en phase de maintenance et l'application étant en production, je suis maintenant passé à temps plein chez BP1818. Comme nous l'avons déjà dit au début de ce rapport, cette banque est une filiale de Natixis Bank qui, étant satisfait du travail effectué, souhaite voir de nouveaux projets fleurir. Ainsi, Natixis Bank Luxembourg souhaite elle aussi mettre en place une application web très semblable au fronting digital de BP1818, projet qui pourrait être réalisé avec l'équipe actuelle qui connait bien le produit.
	
\end{changemargin}